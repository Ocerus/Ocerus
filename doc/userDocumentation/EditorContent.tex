\begin{description}
  \item[Namespaces:] Editor
  \item[Headers:] CreateProjectDialog.h, EditorGUI.h, EditorMenu.h, EditorMgr\-.h, EntityWindow.h, HierarchyWindow.h, KeyShortcuts.h, LayerWindow.h, PrototypeWindow.h, ResourceWindow\-.h and e\-di\-tors of properties
  \item[Source files:] CreateProjectDialog.cpp, EditorGUI.cpp, EditorMenu.cpp,\\ EditorMgr.cpp, EntityWindow.cpp, HierarchyWindow.cpp, KeyShortcuts.cpp, LayerWindow.cpp, PrototypeWindow\-.cpp, ResourceWindow\-.cpp and editors of properties
  \item[Classes:] CreateProjectDialog, EditorGUI, EditorMenu, EditorMgr, EntityWindow, HierarchyWindow, KeyShortcuts, LayerWindow, PrototypeWindow, ResourceWindow and editors of properties
  \item[Libraries used:] CEGUI
\end{description}

\section{Purpose of the editor}

The editor is used for an easy creation of a new game based on this engine. It provides a well-arranged graphic user interface for managing everything from the whole project to each entity in several scenes. The main advantage is that every change made to a scene is immediately visible in the game window where the game action can be started anytime.

In the following sections the usage of the editor to make a new game as well as the classes providing the editor will be described. In the last section there is a small glossary of used terms.

\section{Using the editor}
\label{sec:editor-using}

In this section the editor user interface will be described. First the managing of whole projects will be introduced as well as managing of their scenes. Then working with entities will be described including their organization and editing with editor tools. Finally the rest of editor features will be revealed.

\subsection{Managing projects}

After the editor application starts you should either create a new project, or open an old one. For the creation of a new project choose \emph{File}$\rightarrow$\emph{Create Project} from the main menu. To the displyed dialog fill the project name and choose the location where the project will be stored and click to the \emph{Ok} button. Then the new project with the default configuration and the selected name will be created.

For opening of an old project choose \emph{File}$\rightarrow$\emph{Open Project} from the main menu. Then find and select the project directory and click to the \emph{Ok} button. If everything is alright the project will be loaded and the default scene will be opened, otherwise the error message will be displayed.

When the project is finished the game can be deployed for running without the editor on computers without this game engine installed. This can be done by selecting a target platform from the \emph{File}$\rightarrow$\emph{Deploy Project} submenu list. Then a directory where game files will be stored should be selected in the shown dialog. Finally a message box indicates whether the deploy was successful.

It is also possible to close project by the \emph{File}$\rightarrow$\emph{Close Project} choice or quit application by clicking on the \emph{File}$\rightarrow$\emph{Quit} menu item.

\subsection{Managing scenes}

Every project contains one or more scenes which are loaded separately, have own lists of entities (game objects) and layers but a common list of entity prototypes (templates) and resources.

For creating a new scene choose \emph{Scene}$\rightarrow$\emph{New Scene} from the main menu, select the location of it, type its name and click to the \emph{Ok} button. For opening another scene choose \emph{Scene}$\rightarrow$\emph{Open Scene} from the main menu and select the desired scene from the list. There are also options for saving and closing the current scene in the main menu.

\subsection{Editor windows}

Every scene consists of many entities that consist of components and that can be organized in the hierarchy, in the layers, by the position in the scene and by the prototype they are linked to.

The window with the \emph{Hierarchy} caption is used to manage entities in the entity tree (one entity can have one parent entity and none or many child entities). It is possible to change a parent of an entity by drag and drop it to another entity that should be its new parent or after the last entity to get it to the top of the hierarchy. For moving an entity up or down on the same hierarchy level use the options in the popup menu that can be activated by right mouse button click on the required entity.

Every entity with the \emph{Transform} component is presented in the layer manager (window with the \emph{Layers} caption) in its layer. The layers are sorted from the foreground to the background so entities in the first layer are rendered after (are before) entities in the second one. To show the list of the entities in some layer click on the \emph{+} sign before its name, to hide it click on the \emph{-} sign at the same place. To show/hide all entities in some layer in the viewports click on the picture with an eye behind its name. To make some layer active (new entities are created to it and the entities in viewport are selected from it) doubleclick to its name that will become bolder. To move the entity to another layer drag and drop it to its name or drag the layer name to the \emph{Layer} property in the entity editor window (see more in the subsection \ref{sub:editor-entities}). To move the layer in front of/behind another one drag and drop it to its name. The last two actions can be done also by selecting the corresponding option in the popup menu activated on the layer name as well as renaming or removing it. To add a new layer click on the \emph{New Layer} menu item in the popup menu and fill a new name to the prompt dialog.

Every entity with the \emph{Transform} component and the \emph{Sprite} or the \emph{Model} component is visible in the editor viewport (the bottom window) on its current position. In this viewport it is possible to select entities and manipulate with them by editor tools (see the subsection \ref{sub:editor-tools}) or to zoom in or zoom out the editor camera by the mouse wheel for showing various parts of the scene. In the game viewport (the top window) the entities are shown as it will look in the game (see the subsection \ref{sub:editor-tools}).

In the window with the \emph{Prototypes} caption entity prototypes of the current project are listed. Further information for managing them is mentioned in the subsection \ref{sub:editor-prototypes}.

\subsection{Managing entities}
\label{sub:editor-entities}

There are several ways to create a new entity. Choose \emph{Edit}$\rightarrow$\emph{New Entity} from the main menu for creating an entity with the \emph{Transform} component at the active layer, in the center of the editor viewport and at the top of the entity hierarchy. Choosing the \emph{Add entity} option in the popup menu activated on some entity in the hierarchy window will do the same except this entity will be a parent of the new one. Another possibility is to create an entity from a prototype by popup menu on the prototype window or dragging an item from the prototype list to the hierarchy list or to the editor viewport. It creates an entity with the same components as the chosen prototype has and which is linked to it so it shares some property values with it (see the subsection \ref{sub:editor-prototypes}).

For editing entity properties choose an entity from the hierarchy window, from the layer window or from the editor viewport (only entities in the active layer can be selected) by the left mouse button click. Then the window with \emph{Entity Editor} caption will be filled with general information about the current entity (an ID, a name, a tag and a linked prototype) and with all of its components with their properties. A new component can be added to the entity by choosing the desired one from the \emph{Edit}$\rightarrow$\emph{New Component} list in the main menu or from the similar list in the popup menu activated on the current entity in the hierarchy window. An existing component can be removed from the entity by clicking on the button in the top right corner of it in the entity editor window if it is possible (components can be dependent on each other).

Every writable property of the current entity can be changed in the entity editor window by the property editor displayed near the property name. For every property type there can be a different editor. For example the vector property has two edit boxes for x and y coordinates, while the boolean property can be changed by (un)checking the check box. The properties representing a resource (a texture, a model, a script etc.) cannot be edited directly but the resource file name must be dragged from the resource window (with the \emph{Resources} caption) and dropped on the corresponding edit box in the entity editor. The array properties have several editors for each array item under them that can be added by the \emph{Add} button and removed by its \emph{RM} button. The changes in the array property must be confirmed or reverted by the \emph{Save} or \emph{Rvrt} button.

The current entity can be duplicated or deleted by selecting the \emph{Edit}$\rightarrow$\emph{Du\-pli\-ca\-te Entity} or \emph{Edit}$\rightarrow$\emph{Delete Entity} from the main menu or the corresponding option of the hierarchy window popup menu. More than one entity from the active layer can be selected by holding the \emph{Shift} key and dragging over them in the editor viewport. This selection can be duplicated or deleted by selecting the \emph{Edit}$\rightarrow$\emph{Duplicate Selected Entities} or \emph{Edit}$\rightarrow$\emph{Delete Selected Entities} from the main menu.

\subsection{Managing prototypes}
\label{sub:editor-prototypes}

An entity can be linked to a prototype with the same components, which means it shares some of prototype's properties. This can be done by dropping the prototype dragged from the prototype window to the corresponding cell in the general information section of the entity editor window where the name and ID of the current prototype are shown and where it is also possible to unlink them by clicking on the \emph{RM} button. A new prototype can be created from the current entity by selecting \emph{Edit}$\rightarrow$\emph{Create Prototype} or by the \emph{Add Prototype} option of the prototype window popup menu where it is also possible to delete another one by the \emph{Delete Prototype} option.

Components and properties of a prototype can be also edited in the entity editor after the selection in the prototype window. The prototype editor looks almost the same as the entity one but before each property there is a check box which indicates whether it should be shared or not. If some property is shared the change of its value in the prototype is propagated to all entities linked to it and it is not possible to edit it in the linked entities (this is indicated by a lock picture behind the property editor).

\subsection{Editor tools}
\label{sub:editor-tools}

There is an editor toolbar at the top of the screen with buttons for running action and easier manipulating with entities. The first three buttons serve for resuming, pausing and restarting the game action. When the action is running the entities are affected by physics and scripts so it is possible to see in the game viewport how the scene will look like in the real game. The action can be paused and resumed by the first two buttons while the third button returns the scene to the state before the first resuming of the action (including all changes made to the entities).

The last four buttons serve for moving, rotating and scaling selected entities. Just select the tool (by pressing the corresponding button) and one or more entities in the editor viewport, hold the left mouse button and move with the mouse pointer. The first tool moves the entities, the second rotates them along the z coordinate (orthogonal to the computer screen), the third rotates the entities with the \emph{Model} component along the y coordinate and the last changes the x and y scale.

For opening this file with the user documentation or the file with possible shortcuts or for showing the information about the editor select the corresponding item in the \emph{Help} list in the main menu.

\section{Description of editor classes}

In this section the classes providing the editor will be described. They do not have to be contained in the final distribution of the game if an editor support should not be allowed. First subsection is focused on the logical and graphical managers of the editor whereas the second subsection is about layouts used to built the GUI of the editor.

\subsection{Editor managers}

The \emph{Editor::EditorMgr} class manages the logical part of the editor and it owns an instance of the \emph{Editor::EditorGUI} class that focuses to the GUI of the editor. Both of them have methods called at the loading and the unloading of the editor and also methods for updating logic and drawing in the application loop.

The first mentioned class has methods for managing projects, scene and entities. It manages selecting entities and editing the current one, it reacts on choosing all items in the main menu such as creating a new entity, duplicating and deleting current or selected entities, adding and removing entity components, creating a prototype from a current entity, creating or loading a project or a scene and it solves changing an entity name or an entity property. It also provides a function for all edit tools such as moving, rotating or scaling of a chosen entity and for resuming, pausing and restarting the game action. Finally there are methods for getting reference to all editor windows and for updating them.

The second class compounds all editor layouts such as the game and editor viewports, the resource, prototype and layer windows and the editor menu and initializes and updates them. It directly updates the entity editor window according to the current entity that uses the vertical layout for positioning entity components and various value editors for showing and editing all kinds of entity properties such as strings, vectors, resources and arrays.

\subsection{Writing own value editors}

All value editors derives from the \emph{Editor::AbstractValueEditor} base class. This class has three abstract methods that every value editor must implement and some methods that can help with the implementation. The first is the \emph{AbstractValueEditor::CreateWidget} method with the parameter representing a string that should be used as a prefix for a name of every created component. This method should create a widget for editing a required kind of value and return it. The second is the \emph{AbstractValueEditor::Update} method which is called when the current value of the property should be displayed in the value editor and the last is the \emph{AbstractValueEditor::Submit} method which is called when the value of the property should be updated by the user input to the value editor.

For an easier implementation of value editors the model classes with the \emph{Editor::IValueEditorModel} class as the base class were created. From the accessors of this class value editors should get the name and tool-tip for the edited property, whether the property is valid, read only, element of a list, removable, shareable, shared and also they should be able to remove the property or set whether it is shared if is possible. It is recommended to derive own models from the \emph{Editor::ITypedValueEditor\-Model$<$T$>$} template class that also defines the getter and setter for the value of the edited property.

For example the \emph{Editor::StringEditor} class derives directly from the \emph{Editor::AbstractValueEditor} class and implements a simple editor for properties which values are easily convertible from and to string which contains a label with a property name, an edit box for displaying and editing the value and a remove button if the property is removable (i.e. as a part of an array). It uses a model derived from the \emph{Editor::ITypedValueEditorModel$<$string$>$} class to create a widget, update and submit the value which is specified in the constructor parameter. A useful implementation of this model is the \emph{Editor::StringPropertyModel} class that operates with the \emph{Reflection::Property\-Holder} class from which it gets all necessary information and which it can modify with a new value.

There are another examples of value editors and its models in the code that can help with a creation of further ones such as editors for arrays, resources etc.

\subsection{Used layouts}

There are six layout classes for editor components. All of them have initialization and update methods and callbacks for significant events and load their look from an external XML file and introduce the reactions on events and loading data from the application. The \emph{Editor::Create\-Project\-Dialog} represents the dialog for creating a new project, the \emph{Editor\-::\-Hierarchy\-Window} manages the hierarchy of entities, the \emph{Editor\-::\-Layer\-Window} manages the layers the entities are put to, the \emph{Editor\-::\-Prototype\-Window} manages the prototypes of entities, the \emph{Editor\-::\-Entity\-Window} manages components and properties of entities and the \emph{Editor\-::\-Resource\-Window} manages the resources that the entities can use.

\section{Glossary}
This is a glossary of the most used terms in the previous sections:

\begin{description}
  \item[Project] -- represents one game created in the editor that can be run independently, it is divided to scenes
  \item[Scene] -- represents one part of the game that is loaded at once (i.e. a game level, a game menu etc.)
  \item[Entity] -- represents one object in the scene with specific properties and behavior
  \item[Component] -- a part of an entity which adds some properties and behavior to it
  \item[Prototype] -- a template entity from which is possible to set properties of all linked entities en bloc
  \item[Viewport] -- a part of editor where a scene is shown
\end{description}