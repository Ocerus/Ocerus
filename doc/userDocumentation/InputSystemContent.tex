\begin{description}
  \item[Namespaces:] InputSystem
  \item[Headers:] IInputListener.h, InputActions.h, InputMgr.h, OISListener.h
  \item[Source files:] InputMgr.cpp, OISListener.cpp
  \item[Classes:] InputMgr, IInputListener
  \item[Libraries used:] OIS
\end{description}

\section{Introduction}
Each game must somehow react to the input from the player. It can come from a keyboard, a mouse or a gamepad. To keep things organized handling of those devices were concentrated into a single part of the engine - the Input System.


\section{What does it do}
The needs of games are different, but the ways they want to access the input devices are still the same. Either they want to receive a notification when something interesting happens or they want to poll the device for its current state. The \emph{InputMgr} implements both of the approaches.

The first one is provided via the \emph{IInputListener} interface which you may implement and then register into the \emph{InputMgr} by using its \emph{AddInputListener} method. You'll then receive the notification in callbacks of the interface.

The second approach is supported by multiple methods in \emph{InputMgr} for querying the device state. For example, \emph{IsKeyDown} will tell you if any specific key is currently held down and \emph{GetMouseState} will return a whole bunch of mouse related informations.


\section{Event processing}
To keep things synchronized the event processing is executed in the main game thread. At the beginning of each iteration of the game loop the \emph{Capture} method is called. Inside the events are recognized and then distributed.



%\section{Glossary}
%This is a glossary of the most used terms in the previous sections:
%
%\begin{description}
%  \item[Resource] -- a unit of data the game will be working with as a whole. The data is usually stored in an external device.
%\end{description}


%\begin{thebibliography}{9}                                                                                                
%\bibitem {angelscript}AngelScript documentation, file /AngelScript/index.html
%\end{thebibliography}
