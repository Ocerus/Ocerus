\documentclass[a4paper, 12pt]{report}

\usepackage[USenglish]{babel}
\usepackage[T1]{fontenc}
\usepackage[ansinew]{inputenc}
\usepackage{lmodern} %Type1-font for non-english texts and characters
\usepackage{graphicx}

\newenvironment{titled-itemize}[1]
{
\vspace{5mm}
\noindent\textbf{#1}
\begin{itemize}
}
{
\end{itemize}
}

\begin{document}

\pagestyle{empty} %No headings for the first pages.

\title{Documentation to project Ocerus}
\author{Lukas Hermann, Ondrej Mocny, Tomas Svoboda}
\maketitle

\pagestyle{plain} %Now display headings: headings / fancy / ...

\tableofcontents %Table of contents
\cleardoublepage %The first chapter should start on an odd page.

\chapter{Introduction}

This document serves as both user and design documentation to the project Ocerus. In this opening chapter the purpose of this project will be revealed, the structure of this document and the recommended way of reading it will be presented and finally the project architecture will be described.

\section{Purpose of this project}

The project Ocerus implements a multiplatform game engine and editor for creating simple 2.5D games. The main focus is on editing tools integrated to the engine so every change is immediately visible in the game context and can be tested in a real time. The engine takes care of rendering and physical representation of game entities, easy customization of their behavior by scripts, managing necessary resources and input devices and it provides an easy connection among them for building the whole game. It is easily expandable of new parts from which game entities are built, possible communication among them, data and methods accessible from scripts etc. that is supported by logging, debugging and profiling tools.

\section{How to read this documentation}

Since this product is intended to be used by game developers rather than end users and since it is necessary to understand some design solutions for developing more complex games with this product the user and design documentation are merged into one document. This section should help users of this product with an orientation in this document and tell them where the information they need is written.

The first section that should be read before starting a development of a new project is definitely the section \ref{sec:editor-using} which describes the basic work with the GUI of the editor -- how to create a new project, add a new scene, create entities and manipulate with them etc. For a change of a default entity behavior it is necessary to understand the associated script language syntax (\ref{sec:script-language}) and its functions and methods (\ref{sub:script-entity} and \ref{sub:script-registered}). If some behavior cannot be added by scripts or it would be inefficient new components from which entities are built can be added, which is described in section \ref{sec:entity-extending}.

For adding a game menu or any text to the game the section \ref{sec:gui-creating} about a creating a GUI should be read. For a multilanguage support all texts in the game should not be hardcoded but the string system described in the chapter \ref{chap:string} should be used. If some configuration is needed see the section \ref{sec:config}. When a new native code is added it can be useful to log some information (\ref{sec:logging}), to profile it (\ref{sec:profiling}), to use some helper classes (\ref{chap:utils}) and to access it from scripts (\ref{sub:script-registering}).

\section{Project architecture}

The project Ocerus is logically divided into several relatively independent systems which cooperates with each other. Every system maintains its part of the application such as graphics, resources, scripts etc. and provides it to other ones. In the picture \ref{fig:system-connection} the relations among all systems are displayed with a brief description of what the systems provides to each other.

\begin{figure}[htbp]
	\centering
		\includegraphics[width=1\textwidth]{SystemConnection.pdf}
	\caption{Dependencies among the systems}
	\label{fig:system-connection}
\end{figure}

This project is not coded from the basics but it uses several libraries to allow developers to focus on implementing important improvements for end users and on higher design solutions rather than low level programming of for example the graphic or script engine. All used libraries support many platforms, have free licenses and have been heavily tested in lot of other projects. All of them are used directly by one to three project systems except the library for a unit testing. The library dependence of all systems is displayed in the picture \ref{fig:library-dependence}.

\begin{figure}[htbp]
	\centering
		\includegraphics[width=1\textwidth]{LibraryDependence.pdf}
	\caption{Library dependencies of the project systems}
	\label{fig:library-dependence}
\end{figure}

In this list a brief description of all used libraries is mentioned:

\begin{itemize}
  \item AngelScript -- a script engine with an own language
  \item Boost -- a package of helper data structures and algorithms
  \item Box2D -- a library providing 2D real-time physics
  \item CEGUI -- a graphic user interface engine
  \item DbgLib -- tools for a real-time debugging and crash dumps
  \item Expat -- a XML parser
  \item OIS -- a library for managing events from input devices
  \item OpenGL -- an API for 2D and 3D graphics
  \item RTHProfiler -- an interactive real-time profiling of code
  \item RudeConfig -- a library for managing configure files
  \item SDL -- a tool for an easier graphic rendering
  \item SOIL -- a library for loading textures of various formats
  \item UnitTest++ -- a framework for a unit testing
\end{itemize}

In the following chapters each of the project systems will be described from both user and design view. At the beginning of each chapter there is a section about a purpose of the described system and at the end of most chapters there is a small glossary of terms used in the chapter.

\chapter{Core}

\begin{description}
  \item[Namespaces:] Core
  \item[Headers:] Application.h, Config.h, Game.h, LoadingScreen.h
  \item[Source files:] Application.cpp, Config.cpp, Game.cpp, LoadingScreen.cpp
  \item[Classes:] Application, Config, Game, LoadingScreen
  \item[Libraries used:] Box2D, RudeConfig
\end{description}

\section{Purpose of the core}

The Core namespace is the main part of the whole system. It contains its entry point and other classes closely related to the application itself. Its main task is to initialize and configure other engine systems, invokes their update and draw methods in the main loop and in the end correctly finalize them.

In the following sections the class representing the application as well as the classes corresponding to the application states (loading screen, game) and configuration will be introduced. In the last section there is a small glossary of used terms.

\section{Application}

When the program starts it creates an instance of the class \emph{Core::Application}, initializes it by calling its method \emph{Init} and calls the \emph{RunMainLoop} method which runs until the application is shutdown, then the instance is deleted and the program finishes.

On the initialization of the application the configuration is read (see section \ref{sec:config}) and all engine systems are created and initialized as well as the loading screen and game classes. The state of application is changed to \emph{loading} and the main loop is running until the state is changed to \emph{shutdown}. At the main loop window messages are processed, performance statistic are updated and other engine systems including the game class are loaded (in a \emph{loading} state) or updated and drawn (in a \emph{game} state).

In the application class there are also methods for getting an average and last FPS statistic and methods for showing and hiding a debug console as well as writing message to it. There are also the variables indicating whether the current application instance includes the editor (in a game distribution the editor should be disabled) and whether the editor is currently turned on so the game is running only in a small window instead of a full screen mode.

\section{Game}

The \emph{Core::Game} class manages the most important stuff needed to run the game such as drawing a scene, updating physics and logic of entities, measuring time, handling a game action or resolving an user interaction. Of course it mostly delegates this work to other parts of the engine.

Before the game initialization at the method \emph{Init} a valid render target must be set by method \emph{SetRenderTarget} to know where to draw the game content. This is done for example by the editor when the game is run from it. Then physics, time, an action etc. are initialized and in the \emph{Update} method called in the main loop they are updated.

The drawing of a scene is invoked in the method \emph{Draw}. The render target is cleared, all entities in the current scene are drawn by a renderer and the rendering is finalized.

There are several methods for handling a game action. The action can be paused, resumed and restarted to previously saved position. There is a global timer that measures game time (can be obtain by the method \emph{GetTimeMillis}) when the game is running which is used by other systems such as the script system.

When the action is running physics and logic of entities are updated in the method \emph{Update} which means the corresponding messages are broadcast to all entities before and after the update of the physical engine.

Since the class \emph{Core::Game} registers the input listener to itself there are callbacks where it is possible to react to keyboard and mouse events such as a key or mouse button press/release or a mouse move. The corresponding information such as a current mouse position is available through the callback parameters.

\section{Loading screen}

\section{Configuration}
\label{sec:config}

The \emph{Core::Config} class allows storing a configuration data needed by various parts of the program. Supported data types are strings, integers and booleans and they are indexed by text keys and they can be grouped to named sections.

This class is initialized by a name of the file where data are or will be stored. Although changes to a configuration are saved when the class is being destructed it is possible to force it and get the result of this action by the method \emph{Save}.

There are several getter and setter methods for each data type that get or set data according to a key and a section name. A section parameter is optional, the section named \verb|General| is used as a default. The getter methods has also a default value parameter that is returned when a specific key and section do not exist in a configuration file. It is possible to get all keys in a specific section to a vector with the method \emph{GetSectionKeys} or remove one key (\emph{RemoveKey}) or a whole section (\emph{RemoveSection}).

\section{Glossary}
This is a glossary of the most used terms in the previous sections:

\begin{description}
  \item[Loading screen] -- a screen visible during a loading of the game indicating a loading progress
  \item[Main loop] -- a code where an input from user is handled, an application logic is updated and a scene is drawn in a cycle until an application shut down
  \item[FPS] -- a count of frames per second that are drawn indicates a performance of a game
  \item[Render target] -- a region in a application window where a game content is drawn to
  \item[Configuration data] -- data that parametrizes the application running (i.e. a screen resolution, a game language etc.)
\end{description}

\chapter{Entity system}

\begin{description}
  \item[Namespaces:] EntitySystem, EntityComponents
  \item[Headers:] Component.h, ComponentEnums.h, ComponentHeaders.h, ComponentID.h, ComponentIterators.h, ComponentMgr.h, Com\-po\-nent\-Ty\-pes\-.h, EntityDescription.h, EntityHandle.h, EntityMessage.h, Entity\-Message\-Types.h, EntityMgr.h, EntityPicker.h
  \item[Source files:] Component.cpp, ComponentEnums.cpp, ComponentMgr.cpp, EntityDescription.cpp, EntityHandle.cpp, EntityMessage.cpp, Entity\-Mgr.cpp, EntityPicker.cpp
  \item[Classes:] Component, EntityComponentsIterator, ComponentMgr, EntityDescription, EntityHandle, EntityMessage, EntityMgr, EntityPicker
  \item[Libraries used:] none
\end{description}

\section{Purpose of the entity system}

The entity system creates a common interface for a definition of all game objects such as a game environment, a player character, a camera etc. and their behavior such as a drawing on a screen, an interaction with other objects etc. The object creation is based on a composition of simple functionalities that can be reused in many of them. The advantage of this unified system is an easy creating and editing of new objects from the game editor or from scripts, the disadvantage is a slower access to the object properties and behavior. It cooperates with the other systems like the graphics one for displaying objects or the script one for an interaction from scripts.

In the following sections the system of components and entities will be described as well as the extending of the system which will be probably the first action when creating a new game. In the last section there is a small glossary of used terms.

\section{Components and entities}

Every game object is represented by an entity which is a compound of components that provide it various functionalities. A component can have several properties (and functions) which can be read or written (called) via their getters and setters (or functions themselves) and which are accessible through their unique name. It can also react to sent messages such as an initialization, a drawing, a logic update etc. by its own behavior. Component properties and behaviors are accessible only through an owner entity, so it is possible to read or write a specific property of an entity if it contains a component with this property and it is also possible to send a message to an entity which dispatches it to all its components that can react on it.

\subsection{Components and their manager}

The \emph{EntitySystem::Component} class is a base class for all components used in the entity system. It inherits from the \emph{Reflection::RTTIBaseClass} class which provides the methods for working with RTTI (registering properties and functions of component). It has methods for getting the owner entity, the component type (defined in ComponentEnums.h) and the component property from its name and for posting message to the owner entity. It also introduces methods that should be overridden by specific components used for handling messages and the component creation and destruction (see section \ref{sub:entity-newcomponents}).

The \emph{EntitySystem::ComponentMgr} is a singleton class that manages instances of all entity components in the entity system. Internally it stores mapping from all entities to lists of their components. It provides methods for adding a new component of a certain type to an entity and listing or deleting all or specific components from an entity. For passing all components of an entity the \emph{EntitySystem::EntityComponentsIterator} iterator is used that encapsulates a standard iterator (for example it has the \emph{HasMore} method which returns whether the iterator is at the end of the component list).

\subsection{Entities and their manager}

An entity is represented by the \emph{EntitySystem::EntityHandle} class which sto\-res only an ID of the entity and provides methods that mostly calls corresponding methods of the entity manager with its ID. This class has also static methods that ensures all IDs in the system are unique.

For the creation of one entity the \emph{EntitySystem::EntityDescription} class is used that is basically a collection of component types. There are methods for adding a component type and setting a name and a prototype of the entity. It is also possible to set if the created object will be an instance or a prototype of an entity. Prototypes of entities are used to propagate changes of their shared properties to the instances that are linked to them so it is possible to change properties of many entities at once. Instances must have all components that has their prototype in the same order but they can also have own additional components that must be added after the compulsory ones.

It is possible to send messages to entities so there is the \emph{EntitySystem::EntityMessage} structure that represents them. It consists of the message type defined in EntityMessageTypes.h and the message parameters that are an instance of the \emph{Reflection::PropertyFunctionParameters} class. To add an parameter of any type defined in PropertyTypes.h the \emph{PushParameter} method can be called with a value as first argument or the \emph{operator}$<<$ can be used. There is also a method that checks whether the actual parameters are of the correct types according to the definition of message type (see section \ref{sub:entity-newmessages} for more information).

All entities are managed by the \emph{EntitySystem::EntityMgr} class that stores necessary information about them in maps indexed by their ID. The most of its methods has the entity handle as the first parameter that means it applies on the entity of the ID got from the handle. There are methods for creating entities from an entity description or an XML resource and for destroying them. Another methods manages entity prototypes - it is possible to link/unlink an instance to/from a prototype, to set a property as (non)shared and to invoke an update of instances of a specific prototype. Finally there are methods for getting entity properties even of a specific component (in case of two or more properties of a same name in different components), for posting and broadcasting messages to entities and for adding, listing and removing components of a specific entity.

% dokumentace tridy EntityPicker

\section{Extending the entity system}

The entity system will be probably the first system to extend when creating a new game. It is necessary to create new components with their specific properties and behavior from which new entities can be created in the game editor or from scripts. It is also possible to create new message types that can be sent to entities when it is needed to inform about new different events.

\subsection{Creating new components}
\label{sub:entity-newcomponents}

There are several steps that leads to creating a new component. First it is needed to create a class \emph{ComponentName} (to be replaced by a real component name) which publicly inherits from the \emph{Reflection::RTTIGlue$<$Component\-Name, Component$>$} class and which is in the \emph{EntityComponents} namespace. This class should override the \emph{Create}, \emph{Destroy} and \emph{HandleMessage} methods for a custom behavior on creation, destruction and handling messages. In the last method the message structure is got from the first parameter so it is possible to get a message type and message parameters (the \emph{GetParameter(index)} method is used). The method should return \emph{EntityMessage::RESULT\_OK} if the message was processed, \emph{EntityMessage::RESULT\_ IGNORED} if it was ignored or \emph{EntityMessage::RESULT\_ERROR} if an error occured.

The last common method of the class will be the \emph{RegisterReflection} me\-thod that is static and that is called automatically when the application initializes. It should register all properties and functions which the component provides by the \emph{RegisterProperty} and \emph{RegisterFunction} methods inherited from the base class. In case of a property the following information must be specified: a type (from PropertyTypes.h), a name (must be unique in a component), a getter (a constant function that returns value of a same type and has no parameters), a setter (a non-constant function with one constant parameter of a same type and no return value), an access flags (a disjunction of the \emph{Reflection::ePropertyAccess} enumerations) and a comment (will be displayed in the editor). A getter and a setter can be simple functions that return or modify a private member variable or they can do a more complex computing. For example the registration of an integer property with a getter and a setter as methods of the class and with a read and write access from the editor looks like:
\begin{verbatim}
RegisterProperty<int32>("IntProp", &ComponentName::GetIntProp,
    &ComponentName::SetIntProp, PA_INIT | PA_EDIT_READ |
    PA_EDIT_WRITE, "This is an integer property.");
\end{verbatim}

After the creation of the class a new component type must be registered in ComponentTypes.h where a new line like \verb/COMPONENT_TYPE(CT_COMPONENT_/ \verb/NAME, ComponentName)/ should be added. Finally the header file where the component is declared must be added to ComponentHeaders.h, for example \verb|#include "../Components/ComponentName.h"/|. Now the component is ready to be added to entities.

\subsection{Adding new entity message types}
\label{sub:entity-newmessages}

If it is necessary to add a new entity message type that can be posted to entities then the new line to EntityMessageTypes.h must be added. It should look like \verb/ENTITY_MESSAGE_TYPE(MESSAGE_NAME, "void OnMessageName(type1,/
\verb/type2, ...)", Params(PT_TYPE1, PT_TYPE2, ...))/ where the first parameter is an enumeration constant, the second one is a declaration of script function that handles the message and the last one is a definition of message parameters where \verb/PT_TYPE/ is from PropertyTypes.h. In case the message has no parameters the \verb/NO_PARAMS/ macro should be provided as the last parameter.

\section{Glossary}

This is a glossary of the most used terms in the previous sections:

\begin{description}
  \item[Entity property] -- a named pair of a getter and a setter function of a specific type with certain access rights
  \item[Entity function] -- a named link to a function with a \emph{Reflection::Property\-FunctionParameters} parameter and certain access rights
  \item[Entity message] -- a structure that stores a message type from EntityMessageTypes.h and message parameters
  \item[Component] -- a class which has registered functions and properties, that can be read and written via their getters and setters, and which can handle received messages
  \item[Entity] -- a compound of one or more components, that provide specific functionalities, represented by an unique ID, it is possible to post a message to it
  \item[Prototype] -- changes of shared property values of this entity are propagated to the linked entities
\end{description}

\chapter{Gfx system}

\begin{description}
  \item[Namespaces:] GfxSystem
  \item[Headers:]GfxRenderer.h,GfxSceneMgr.h, GfxStructures.h, GfxWindow.h, IGfxWindowListener.h, OglRenderer.h, Texture.h, RenderTarget.h, GfxViewport.h
  \item[Source files:] GfxRenderer.cpp,GfxSceneMgr.cpp, GfxStructures.cpp, GfxWindow.cpp, OglRenderer.cpp, Texture.cpp, GfxViewport.cpp
  \item[Classes:] GfxRenderer, OglRenderer, GfxSceneMgr, GfxWindow, IGfxWindowListener, Texture, GfxViewport
  \item[Libraries used:] SDL, OpenGL, SOIL
\end{description}

\section{Introduction}

One of the main goals of Ocerus was that it should be as platform-independent as possible. The most troublesome part of the engine in this matter is the graphics which in Ocerus is done by the GfxSystem.


\section{Application window}
Creating application window very much depends on used opreating system. In Ocerus, this functionality is implemented by \emph{GfxWindow} with usage of the SDL library.

SDL (Simple DirectMedia Layer) is a free cross-platform multi-media development API used for games, emulators, MPEG players, and other applications. The main advantage is that it supports many operating systems.

It is used for handling window events aswell. Currently, the only event that is being handled is the "change resolution" event. There is a listener implemeted for this event(\emph{IGfxWindowListener}) so every other system that needs to be informed about a resolution change can add its own listener. This feature is used by InputSystem and GUISystem.

Note that SDL also provides features in low-level audio and input management but since audio is not yet implemented and input management is done by more specialized library, the only SDL features used is Ocerus are window management and creating rendering context.

\section{Rendering}
In today's games it's quite neccesary to use hardware accelareted graphics even in 2D. It can speed up the rendering few-times, especially when rotating and blending sprites.

\subsection{DirectX vs.OpenGL}
For accelarated graphics there are basically only two suitable libraries: DirectX and OpenGL. Problem with DirectX is that it is developed by Microsoft and it runs only under Windows OS. Problem with OpenGL is the it doesn't run on X-box. 
OpenGL support is implemented first as X-box is not actual target platform of Ocerus.

\subsection{Rendering managment}
The rendering managment is done by singleton class \emph{GfxRenderer}. Since it is necessary to be able to easily implement both OpenGL and DirectX support, the \emph{GfxRenderer} is abstract class which is therefore inherited by platform-dependent classes. Currently, \emph{OglRenderer} which uses OpenGL is implemented.

Note that \emph{GUISystem} uses its own rendering management.

\subsection{Textures}
For loading textures, SOIL library is used.
SOIL (Simple OpenGL Image Library) is a tiny C library used for uploading textures into OpenGL. It supports most of the common image formats.

Loading textures is implemented in class \emph{Texture}. It inherits from  \emph{Resource} class, see \emph{ResourceSystem}. 

Since SOIL depends on OpenGL, inherited class \emph{OglTexture} is introduced. It is the same concept as with the \emph{GfxRenderer/OglRender}.
 
\subsection{Spatial partitioning}
Not yet implemented.

\subsection{Rendering entities}
Not final version ready.


%\section{Glossary}
%This is a glossary of the most used terms in the previous sections:
%
%\begin{description}
%  \item[viewport] -- part of application window to where it is possible to render scene.
%\end{description}


%\begin{thebibliography}{9}                                                                                                
%\bibitem {angelscript}AngelScript documentation, file /AngelScript/index.html
%\end{thebibliography}


\chapter{GUI system}

\begin{description}
  \item[Namespaces:] GUISystem
  \item[Headers:] CEGUICommon.h, CEGUIForwards.h, CEGUIResource.h, FolderSelector.h, GUIConsole.h, GUIMgr.h, MessageBox.h, PopupMgr.h, Prompt\-Box.h, ResourceProvider.h, ScriptProvider.h, TabNavigator.h, VerticalLayout.h, ViewportWindow.h
  \item[Source files:] CEGUIResource.cpp, FolderSelector.cpp, GUIConsole.cpp, \\GUIMgr.cpp, MessageBox.cpp, PopupMgr.cpp, PromptBox.cpp, Re\-sourceProvider.cpp, ScriptProvider.cpp, TabNavigation.cpp, Ver\-ti\-cal\-Layout.cpp, ViewportWindow.cpp
  \item[Classes:] CEGUIResource, FolderSelector, GUIConsole, GUIMgr, MessageBox, PopupMgr, PromptBox, ResourceProvider, ScriptCallback, ScriptProvider, VerticalLayout, ViewportWindow
  \item[Libraries used:] CEGUI
\end{description}

\section{Purpose of the GUI system}

The GUI system provides creating and drawing a graphic user interface based on the CEGUI library for the editor and the game itself. It also manages an user interaction with GUI elements, element layouts, viewports and GUI console.

In the following sections the connection to the engine will be described as well as the method of creating an own GUI. In the last section there is a small glossary of used terms.

\section{Connection to the engine}

In this section all necessary parts of GUI system will be introduced with their connection to the other parts of engine.

\subsection{GUI manager}

The main class of GUI system is the \emph{GUISystem::GUIMgr} which is a connector for drawing, input handling and resource providing between the CEGUI library and the engine. During creation it creates a CEGUI renderer, connects the resource manager with the CEGUI system via the \emph{GUISystem::Resource\-Provider} class (see section \ref{sec:gui-resources}), provides itself as a input and screen listener and creates GUI console which is then accesible via the \emph{GUIMgr::GetConsole} method (see section \ref{sec:gui-console}). On initialization (\emph{GUIMgr::Init}) it loads necessary GUI resources (schemes, imagesets, fonts, layouts, looknfeels) and creates a root window.

For loading a root layout from a file the method \emph{GUIMgr::LoadRootLayout} is provided with only one parameter specifying a name of a file where a layout is defined. This method calls the \emph{GUIMgr::LoadWindowLayout} which is a common method for loading a window layout from a file that also provides a translation of all texts via the string manager. There are also methods for unloading and getting the current root layout. For more information about layout see section \ref{sec:gui-layouts}.

Sometimes it is needed to disconnect a callback function from an event such as a mouse click on a button. Since the CEGUI library crashes when an event is disconnected in its callback the \emph{GUIMgr::DisconnectEvent} method is introduced which adds the event to the list of events that should be disconnected and the real disconnection will process after calling the method \emph{GUIMgr::ProcessDisconnectedEventList} in the application main loop.

There are two more methods called in the application main loop. First the \emph{GUIMgr::Update} updates time of GUI system, then the \emph{GUIMgr::Render\-GUI} draws the whole GUI. There are several input callback methods that converts an OIS library representation of keyboard and mouse events to a CEGUI one and forwards them to the CEGUI library. It is possible to get the currently processing input event by the \emph{GUIMgr::GetCurrentInputEvent} method. There is also a callback method for a resolution change that forwards this information to the CEGUI library too.

\subsection{GUI resources}
\label{sec:gui-resources}

A GUI resource is represented by the \emph{GUISystem::CEGUIResource} class. Since the CEGUI library is not designed to allow an automatic resource unloading and reloading on demand this class only loads raw data by the resource manager and after providing them to the CEGUI library by the \emph{CEGUIResource::GetResource} method it unloads them.

When the CEGUI library needs a resource it calls an appropriate method of a resource provider class provided on an initialization of the library. In this engine it is the \emph{GUISystem::ResourceProvider} class and its method \emph{ResourceProvider::loadRawDataContainer} that gets the resource from the resource manager and forwards its data to the library.

\subsection{Layouts}
\label{sec:gui-layouts}

A GUI layout defines a composition of GUI elements including their properties such as position, size and content and their behavior. Their properties can be defined in an external XML file but more dynamic compositions need also a lot of a code support, their behavior can be defined in an external script file but more complicated reactions have to be also native coded.

As an example that can be used both in the editor and in the game the \emph{GUISystem::MessageBox} class was created which provides a modal dialog for informing the user or for asking the user a question and receiving the answer. The basic layout with all possible buttons is specified in an XML file which is loaded in a class constructor where these buttons are mapped to the correspondent objects and displayed according to a message box type. Setting of a message text (\emph{MessageBox::SetText}) also changes a static text GUI element specified in an XML file. The behavior after the user clicks to one of buttons is defined by a callback function that can be registered by the \emph{MessageBox::RegisterCallback} method and that gets the kind of the chosen button and the ID specified in a constructor parameter. For an easier usage there is the global function \emph{GUISystem::ShowMessageBox} that takes all necessary parameters (a text, a kind of a message box, a callback and an ID) and creates and shows an appropriate message box. A similar concept have the \emph{PromptBox} class providing a modal dialog that asks for a text input from the user and the \emph{GUISystem::FolderSelector} class providing a modal dialog for selecting a folder.

Another example is the \emph{GUISystem::VerticalLayout} class that helps to keep GUI elements positioned in a vertical layout and automatically repositions them when one of them changes its size. In its constructor the container in which all child elements should be managed is specified, then the \emph{VerticalLayout::AddChildWindow} method is used for adding them. It is also possible to set a spacing between them and there is a method for updating a layout. It is obvious that this layout is defined without any XML file. For more information about creating own layouts see section \ref{sec:gui-creating}.

\subsection{Viewports}

The \emph{GUISystem::ViewportWindow} class represents a viewport window with a frame where a scene is rendered by the graphic system. For defining a position, an angle and a zoom of a view of a scene which will be displayed in the viewport a camera in form of an entity with a camera component must be set by the \emph{ViewportWindow::SetCamera} method. It is possible to define whether the viewport allows a direct edit of a view and displayed entities by the \emph{ViewportWindow::SetMovableContent} method. For example in the editor there are two viewports -- in the bottom one the scene can be edited whereas in the top one the result is only shown. The method \emph{ViewportWindow::AddInputListener} registers the input listener so any class can react to mouse and keyboard actions done in the viewport when it has been activated by the method \emph{ViewportWindow::Activate}.

\subsection{Popup menus}

The \emph{GUISystem::PopupMgr} class provides methods creating (\emph{Popup\-Mgr\-::\-Create\-Popup\-Menu} / \emph{PopupMgr::CreateMenuItem}) and destroying (\emph{PopupMgr::DestroyPopupMenu} / \emph{PopupMgr::DestroyMenuItem}) popup menu and its items as well as showing (\emph{PopupMgr::ShowPopup}) and hiding (\emph{PopupMgr::HidePopup}) it and it also cares about calling a proper callback when a menu item is clicked on. The callback method is provided to the manager when the popup menu is being opened by the \emph{PopupMgr::ShowPopup} method.

\subsection{GUI console}
\label{sec:gui-console}

The \emph{GUISystem::GUIConsole} class manages the console accessible both in the game and in the editor. The console receives all messages from the log system via the method \emph{GUIConsole::AppendLogMessage} and shows those ones which have an equal or higher level than previously set by the \emph{GUIConsole::SetLogLevelTreshold} method. In addition the user can type commands to the console prompt line which are sent to the script system as a body of a method without parameters that is immediately built and run and if there is a call of the \verb/Print/ function its content will be printed to the console via \emph{GUIConsole::AppendScriptMessage} method. For better usage of the console a history of previously type commands is stored and can be revealed by up and down arrows. The console can be shown or hide by \emph{GUIConsole::ToggleConsole} method.

\section{Creating GUI}
\label{sec:gui-creating}

This section focuses on creating a GUI to the game. First there is a description of a layout definition, then the way how handle events is introduced and finally the connection of these two parts in the GUI layout component is described.

\subsection{Defining a layout}

The GUI layout for the game should be defined in a file with the \emph{.layout} extension and the XML internal structure. Since the CEGUI library is used for the GUI system a layout must fulfill its specification which is widely described in its documentation \cite{cegui} therefore there is only a brief description it the following paragraphs.

The \verb/<GUILayout>/ element is the root element in layout XML files that must contain a single \verb/<Window>/ element representing the root GUI element. The \verb/<Window>/ element must have the \verb/Type/ attribute which specifies the type of window to be created. This may refer to a concrete window type, an alias, or a falagard mapped type (see the next paragraph). It can have also the \verb/Name/ attribute specifying a unique name of the window. This element may contain \verb/<Property>/ elements with the \verb/Name/ and \verb/Value/ attributes used to set properties of the window, \verb/<Event>/ elements with the \verb/Name/ and \verb/Function/ attributes used to create bindings between the window and script functions (see the next subsection) and another \verb/<Window>/ elements as its child windows. For supported window types, properties and events see the CEGUI documentation.

For defining connections between a physical window type and a window look (rendering, fonts, imagesets) a scheme file is used which is another XML file with a specific structure. Its root element is the \verb/GUIScheme/ one that can contain any number of for example \verb/<Imageset>/, \verb/<Font>/, \verb/<LookNFeel>/, \verb/<WindowAlias>/ or \verb/<FalagardMapping>/ elements. For concrete usage of these elements as well as building a GUI layout with own fonts, images etc. see the CEGUI documentation.

The only specific feature of writing GUI layouts for this game engine is the translation of a text in the \verb/Text/ and \verb/Tooltip/ properties if it is surrounded by \$ characters (i.e. \verb/Text=$text$/). This text is used as a key and the word \verb/GUI/ as a group for the string manager and it is replaced by the result according to the current language when the layout is loaded to the engine.

\subsection{Event handling}

Events generated by an interaction of user with GUI elements (mouse clicking, key pressing, etc.) can be handled by script functions written to a script module. If a layout is connected with a script module via a GUI layout component (see the next section) every occurrence of an event specified in the \verb/Name/ attribute of the layout element \verb/<Event>/ will be followed by calling a script function with the name equal to the \verb/Function/ attribute in the same tag with a \verb/Window@/ parameter holding a reference to the window which the element is child of.

For example assume this is a part of a layout

\begin{verbatim}
<Window Type="CEGUI/PushButton" Name="Continue">
  ...
  <Event Name="MouseClick" Function="ContinueClick">
  ...
</Window>
\end{verbatim}

\noindent and this is a part of a script module

\begin{verbatim}
void ContinueClick(Window@ window)
{
  Println(window.GetName());
}
\end{verbatim}

\noindent connected via a GUI layout component contained in any entity in the scene then every click on the button named \verb/Continue/ will print the message \verb/Continue/ to the log.

\subsection{GUI layout component}

The GUI layout component serves as a connection of one GUI layout file and one script module with functions that serves as callbacks to events of GUI elements described in the layout. Beside of these two properties there are the property \verb/Visible/ indicating whether the layout is visible, the property \verb/Enabled/ indicating whether layout elements reacts on an user input and the property \verb/Scheme/ referring to the GUI scheme file that is loaded before the layout file.

On its initialization it loads the scheme and the layout to the GUI system and it calls script functions with the declarations \verb/void OnInit()/ and \verb/void OnPostInit()/ in the specified module. Every time it receives the \emph{UPDATE\_LOGIC} message it calls a script function with the declaration \verb/void OnUpdateLogic(float32)/ where it is possible to update all information displayed by GUI elements (use the global function \verb/Window@ GetWindow/ \verb/(string)/ with the name specified in the layout file to get the reference to the window).

To show the GUI layout just add this component to any entity in the scene (or create a new one) and set the \verb/Visible/ property to true. The GUI elements are displayed over all entities and are not affected by cameras nor layers.

\section{Glossary}
This is a glossary of the most used terms in the previous sections:

\begin{description}
  \item[GUI] -- a graphic user interface
  \item[GUI element] -- an element from which the whole GUI is created such as label, edit box, list etc.
  \item[Layout] -- a composition of GUI elements including definition of their properties and behavior
  \item[Viewport] -- a GUI element where a scene can be rendered by the graphic system
  \item[GUI console] -- a window where log and script messages immediately appears and which allows an input of script commands
  \item[GUI event] -- a significant situation made by the user input such as a mouse click or a keyboard press
  \item[Callback] -- a function or method that is called as a reaction to a GUI event
  \item[Scheme] -- a definition of connections between a physical window type and a window look
\end{description}

\chapter{Editor}

\begin{description}
  \item[Namespaces:] Editor, GUISystem
  \item[Headers:] EditorGUI.h, EditorMenu.h, EditorMgr.h, FolderSelector.h, LayerMgrWidget.h, PopupMenu.h, PrototypeWindow.h, ResourceWindow\-.h and editors of properties
  \item[Source files:] EditorGUI.cpp, EditorMenu.cpp, EditorMgr.cpp, FolderSelector.cpp, LayerMgrWidget.cpp, PopupMenu.cpp, PrototypeWindow\-.cpp, ResourceWindow.cpp and editors of properties
  \item[Classes:] EditorGUI, EditorMenu, EditorMgr, FolderSelector, LayerMgrWidget, PopupMenu, PrototypeWindow, ResourceWindow
  \item[Libraries used:] CEGUI, SDL
\end{description}

\section{Purpose of the editor}

The editor is used for an easy creation of a new game based on this engine. It provides a well-arranged graphic user interface for managing everything from the whole project to each entity in several scenes. The main advantage is that every change made to a scene is immediately visible in the game window where the game action can be started anytime.

In the following sections the usage of the editor to make a new game as well as the classes providing the editor will be described. In the last section there is a small glossary of used terms.

\section{Using the editor}
\label{sec:editor-using}

\subsection{Managing projects}

\subsection{Managing scenes}

\subsection{Managing entities}

\subsection{Editor tools}

\section{Description of editor classes}

In this section the classes providing the editor will be described. They do not have to be contained in the final distribution of the game if an editor support should not be allowed. First subsection is focused on the logical and graphical managers of the editor whereas the second subsection is about layouts used to built the GUI of the editor.

\subsection{Editor managers}

The \emph{Editor::EditorMgr} class manages the logical part of the editor and it owns an instance of the \emph{Editor::EditorGUI} class that focuses to the GUI of the editor. Both of them have methods called at the loading and the unloading of the editor and also methods for updating logic and drawing in the application loop.

The first mentioned class manages the editing of entities. It manages selecting entities and editing the current one, it reacts on choosing all items in the entity submenu such as creating a new entity, duplicating and deleting current or selected entities, adding and removing entity components or creating a prototype from a current entity and it solves changing an entity name or an entity property. It also provides a function for all edit tools such as moving, rotating or scaling of a chosen entity and for resuming, pausing and restarting the game action. Finally there are methods handling all popup menus.

The second class compounds all editor layouts such as the game and editor viewports, the resource, prototype and layer windows and the editor menu and initializes and updates them. It directly updates the entity editor window according to the current entity that uses the vertical layout for positioning entity components and various value editors for showing and editing all kinds of entity properties such as strings, vectors, resources and arrays.

\subsection{Writing own value editors}

All value editors derives from the \emph{Editor::AbstractValueEditor} base class. This class has three abstract methods that every value editor must implement and some methods that can help with the implementation. The first is the \emph{AbstractValueEditor::CreateWidget} method with the parameter representing a string that should be used as a prefix for a name of every created component. This method should create a widget for editing a required kind of value and return it. The second is the \emph{AbstractValueEditor::Update} method which is called when the current value of the property should be displayed in the value editor and the last is the \emph{AbstractValueEditor::Submit} method which is called when the value of the property should be updated by the user input to the value editor.

For easier implementation of value editors the model classes with the \emph{Editor::IValueEditorModel} class as the base class were created. From the accessors of this class value editors should get the name and tool-tip for the edited property, whether the property is valid, read only, element of a list or removable and also they should be able to remove the property if is possible. It is recommended to derive own models from the \emph{Editor::ITypedValueEditor\-Model$<$T$>$} template class that also defines the getter and setter for the value of the edited property.

For example the \emph{Editor::StringEditor} class derives directly from the \emph{Editor::AbstractValueEditor} class and implements a simple editor for properties which values are easily convertible from and to string which contains a label with a property name, an edit box for displaying and editing the value and a remove button if the property is removable (i.e. as a part of an array). It uses a model derived from the \emph{Editor::ITypedValueEditorModel$<$string$>$} class to create a widget, update and submit the value which is specified in the constructor parameter. A useful implementation of this model is the \emph{Editor::StringPropertyModel} class that operates with the \emph{Reflection::Property\-Holder} class from which it gets all necessary information and which it can modify with a new value.

There are another examples of value editors and its models in the code that can help with a creation of further ones such as editors for arrays, resources etc.

\subsection{Used layouts}

There are several layouts implemented for the editor that can be used as examples of a creation of an own layout. The \emph{GUISystem::FolderSelector} class represents the layout for selecting a folder and has methods for showing/hiding it, registering callback for the Ok/Cancel pushing and getting the current and selected folder. The \emph{Editor::PopupMenu} class manages all popup menus in the editor and has methods for opening/closing it, for initializing it and for handling all menu choices.

There are three layout classes for editor components. All of them have initialization and update methods and callbacks for significant events. All of them loads their look from an external XML file and introduces the reactions on events and loading data from the application. The \emph{Editor::LayerMgrWid\-get} manages the layers the entities are put to, the \emph{Editor::PrototypeWindow} manages the prototypes of entities and the \emph{Editor::ResourceWindow} manages the resources that the entities can use.

\section{Glossary}
This is a glossary of the most used terms in the previous sections:

\begin{description}
  \item[Project] -- represents one game created in the editor that can be run independently, it is divided to scenes
  \item[Scene] -- represents one part of the game that is loaded at once (i.e. a game level, a game menu etc.)
  \item[Entity] -- represents one object in the scene with specific properties and behavior
\end{description}

\chapter{Input system}

\begin{description}
  \item[Namespaces:] InputSystem
  \item[Headers:] IInputListener.h, InputActions.h, InputMgr.h, KeyCodes.h, OISListener.h
  \item[Source files:] InputActions.cpp, InputMgr.cpp, OISListener.cpp
  \item[Classes:] InputMgr, IInputListener, OISListener
  \item[Libraries used:] OIS
\end{description}

\section{Purpose of the input system}

Since each game must somehow react to the input from the player, the input system was implemented which can handle keyboard and mouse events and forward them to other systems.

In the following section the way of receiving and distributing input events is described.

\section{Input manager}
The needs of games are different, but the ways they want to access the input devices are still the same. Either they want to receive a notification when something interesting happens or they want to poll the device for its current state. The \emph{InputSystem::InputMgr} class implements both of the approaches.

The first one is provided via the \emph{InputSystem::IInputListener} interface which should be implemented and then registered by using the \emph{InputMgr::\-AddInputListener} method for receiving the notification in callbacks of the interface.

The second approach is supported by multiple methods for querying the device state. The \emph{InputMgr::IsKeyDown} method returns whether any specific key is currently held down while the \emph{InputMgr::IsMouseButtonPressed} method does the same with the currently pressed mouse button. Finally the \emph{InputMgr::GetMouseState} returns a whole bunch of mouse related information such as a cursor and wheel position or pressed buttons.

To keep things synchronized the event processing is executed in the main game thread. At the beginning of each iteration of the game loop the \emph{InputMgr::Capture} method is called where the events are recognized and then distributed to the listeners.

The \emph{InputSystem::InputMgr} class is only a proxy class which calls methods of the implementation specific class \emph{InputSystem::OISListener} which implements the \emph{OIS::MouseListener} and \emph{OIS::KeyListener} interfaces from the OIS library used for the platform independent input management.


%\section{Glossary}
%This is a glossary of the most used terms in the previous sections:
%
%\begin{description}
%  \item[Resource] -- a unit of data the game will be working with as a whole. The data is usually stored in an external device.
%\end{description}


%\begin{thebibliography}{9}                                                                                                
%\bibitem {angelscript}AngelScript documentation, file /AngelScript/index.html
%\end{thebibliography}


\chapter{Log system}

\begin{description}
  \item[Namespaces:] LogSystem
  \item[Headers:] Logger.h, LogMacros.h, LogMgr.h, Profiler.h
  \item[Source files:] Logger.cpp, LogMgr.cpp, Profiler.cpp
  \item[Classes:] Logger, LogMgr, Profiler
  \item[Libraries used:] RTHProfiler
\end{description}

\section{Purpose of the log system}

The log system manages internal log messages that are used to provide information about application processes useful to debug the whole project. These messages can have various levels of a severity (from trace and debug messages to errors) and it is possible to set the minimal level of messages to show (i.e. only warning and errors). Another function of the log system is managing a real-time ingame profiling useful for a location of the most time critical parts of the project which can leads to effective optimization.

In the following sections the process of logging messages will be described as well as using a profiler. In the last section there is a small glossary of used terms.

\section{Logging messages}
\label{sec:logging}

The main class responsible for logging messages is the \emph{LogSystem::LogMgr}. At the application start it is initialized with a name of the file to which the messages are also written. There is only one method which logs a message of a certain severity to the file and consoles if exist. This method should not be used directly but via the class \emph{LogSystem::Logger}.

The lifetime of the \emph{Logger} should be only one code statement and it represents one message. In the constructor a level of the message and whether to generate a stack trace are specified. Then a sequence of the operator $<<$ is used to build the message from strings, numbers and other common types (any user type can be supported by specifying an own operator $<<$ overloading). At the end of the statement the destructor is automatically called (if the instance is not assigned to a variable) that called the \emph{LogMgr}'s method with the built message.

For an even easier logging of messages log macros are defined for every supported level of severity, adding the information about the file and the line where it is logged from in case of error or warning message. Thanks to macros it is possible to define the minimum level of severity that should be logged at the compile time so the messages with a lower level are even not compiled to the final program which saves time and memory. In the table \ref{tab:log-macros} there are the macros associated with the levels of severity.

\begin{table}[htbp]
	\centering
		\begin{tabular}{|l|l|l|l|}
			\hline
			Macro & Level & Stack trace & Additional info \\
			\hline
			ocError & error & yes & yes \\
			ocWarning & warning & no & yes \\
			ocInfo & information & no & no \\
			ocDebug & debug & no & no \\
			ocTrace & trace & no & no \\
			\hline
		\end{tabular}
	\caption{Definitions of log macros from the most severe to the least ones}
	\label{tab:log-macros}
\end{table}

If it is for example necessary to inform that the entity (of which the handle is available) is created the following statement should be written anywhere in the code: \verb/ocInfo << handle << " was created.";/. When the process passed this code and the minimum level of messages to log is lower than information level then for example the following message will be generated: \verb/13:05:18: Entity(25) was created./

\section{Profiling functions}
\label{sec:profiling}

The profiling of a block of a code or a whole function is really easy. First the \verb/USE_PROFILER/ preprocessor directive must be globally defined, the class \emph{LogSystem::Profiler} must be initialized at the start of the application and its method \emph{Update} must be called in each application loop.

Then anywhere in a code the \verb/PROFILE(name)/ macro can be typed where the parameter \verb/name/ is used for identification of the corresponding results (the abbreviation \verb/PROFILE_FNC()/ uses the current function name). From that line the profiler will start measuring time and it will stop at the end of the current block or function.

Finally when the application runs a call of \emph{Profiler}'s method \emph{Start} (which can be invoked with a keyboard shortcut \emph{CTRL+F5}) activates a profiling and a call of its methods \emph{Stop} and \emph{DumpIntoConsole} (another press of \emph{CTRL+F5}) deactivates it and writes results to the text console. In addition it is possible to ask whether the profiler is activate via the method \emph{IsRunning}.

\section{Glossary}
This is a glossary of the most used terms in the previous sections:

\begin{description}
  \item[Log message] -- a text describing a specific action or an application state occurred at a certain time with a defined severity
  \item[Level of severity] -- an importance of a message to the application process
  \item[Stack trace] -- a list of functions that the current statement is called from right now
  \item[Logging] -- tracking the code execution by writing log messages to a console and a file
  \item[Console] -- a window where log messages immediately appears
  \item[Profiling] -- measuring a real time of execution of a specific code
\end{description}

\chapter{Resource system}

\begin{description}
  \item[Namespaces:] ResourceSystem
  \item[Headers:] IResourceLoadingListener.h, Resource.h, ResourceMgr.h, Re\-sour\-ce\-Ty\-pes.h, UnknownResource.h, XMLOutput.h, XMLResource.h
  \item[Source files:] Resource.cpp, ResourceMgr.cpp, ResourceTypes.cpp, XML\-Out\-put.cpp, XMLResource.cpp
  \item[Classes:] Resource, ResourceMgr, ResourcePtr, UnknownResource, XML\-Out\-put, XMLResource
  \item[Libraries used:] Boost, Expat
\end{description}

\section{Introduction}

Every game needs to load packs of data from external devices such as the hard drive or network. The data come in blocks belonging together and representing a unit of something we usually call \emph{resource}. Because the games work with loads of resources it is necessary to organize them both in-game and on the disk. Also, the data loaded are usually quite large and it's necessary to free them when possible to save memory.


\section{Resources}
As said before, the resource is a group of data belonging together. In the engine this is represented by the abstract \emph{Resource} class. It contains all the basic attributes of a resource and allows its users to load it or unload it, but the actual implementation depends on the specific type of the resource. For example, an XML file is loaded and parsed in a different way then an OpenGL texture. However, for the user it is never really necessary to know what type of the resource he is working with and so using \emph{Resource} as an abstraction is enough. Only the endpoint subsystem needs to work with the specific type to be able to grab the parsed data out of it. For example, the texture resource can be carried around the system as a common \emph{Resource} until it reaches the graphical subsystem which converts it to the texture resource and grabs the implementation specific texture data out of it.

\subsection{Resource states}
Each resource can be in one of the states described in the table \ref{tab:resource-states} at the given point of time:
\begin{table}[htbp]
	\centering
	{\scriptsize 
	\begin{tabular}{|p{0.13\hsize}|p{0.58\hsize}|}
	  \hline
	  A state & A description \\
		\hline
		Uninitialized & the system does not know about it; it's not registered \\
		Initialized & the system knows about it, but the data are not loaded yet \\
		Unloading & the data are just being unloaded \\
		Loading & the data are just being loaded \\
		Loaded & the resource is fully ready to be used \\
		\hline
	\end{tabular}
	}
	\caption[Possible resource states]{Possible resource states and their description}
	\label{tab:resource-states}
\end{table}

\subsection{Content of the resource}
Once the data for the resource are loaded it must be parsed into the desired format. This can mean a data structure stored directly inside the resource class or just a handle to the data stored in other parts of the system. However, both of these must exist only in single instance in the whole system - in the resource which parsed the data. Otherwise the data could become desynchronized. If the resource was unloaded, a pointer to its data could still exist somewhere. 

For example, the XML resource creates a tree structure for the parsed data and allows its users to traverse the tree. But nowhere in the system exists a pointer to the same tree or any of its parts. Another example is a texture. After it loads the data they are passed into the graphical subsystem which creates a platform specific texture out of it and returns only a the texture handle. The handle is then stored only in the resource.

\subsection{Resource pointers}
Because the resources are passed all around the game system we must somehow prevent any memory leaks from appearing. Doing so is quite easy however - we simply use the shared pointer mechanism. It points to the common abstract resource (\emph{ResourcePtr} class) and to specific resources as well (\emph{ScriptResourcePtr} or \emph{XMLResourcePtr} for example). The abstract resource pointer can be automatically converted to any specific resource pointer, but if the type won't match an assertion fault will be raised to prevent memory corruption. All resource pointers are defined in \verb'ResourcePointers.h'.


\section{Resource manager}
The Resource Manager represented by the \emph{ResourceMgr} class takes care of organizing resources into groups and providing and interface to other parts of the system to control or grab the resources. Coupling resources into groups makes it easier to load or unload a whole bunch of them (for a single level of the game, for example).

\subsection{Hotloading}
To make game development easier, the source of each resource is automatically checked for an update. If it has changed, the resource is automatically reloaded if it was previously loaded. So, for example, if you change a currently loaded texture in an image editor and save it it will be immediately updated in the game.

\subsection{Memory limit}
Because gaming systems have limited memory we must make sure we can limit the memory used by resources. Resources usually take the biggest chunk of memory, so when lowering the memory usage it's best to start here. Hopefully, \emph{ResourceMgr} allows us to define a limit which it will try to keep. When the memory is running out, it will attempt to unload resources which were not used for a long time. These resources will remain in the system and will be ready to be loaded as soon as they are needed. However, there are certain circumstances under which the resources must not be unloaded. An example of such is rendering - no texture can be unloaded until the frame is ready. For this reason the unloading can be temporarily disabled.


\section{Adding custom resource types}
To add a new resource type you must go through two steps. First create a class derived from \emph{Resource} implementing all abstract methods and providing accessors to the parsed data. Then add the corresponding shared pointer to your class into \verb'ResourcePointers.h'. While creating the accessor keep in mind that each of them must call \emph{EnsureLoaded} to make sure the resource is actually loaded before it is to be used!


\section{Currently existing resource types}
To see all currently existing types of resources it's best to head into the doxygen documentation. Locate the \emph{ResourceSystem::Resource} class and see all classes derived from it.


%\section{Glossary}
%This is a glossary of the most used terms in the previous sections:
%
%\begin{description}
%  \item[Resource] -- a unit of data the game will be working with as a whole. The data is usually stored in an external device.
%\end{description}


%\begin{thebibliography}{9}                                                                                                
%\bibitem {angelscript}AngelScript documentation, file /AngelScript/index.html
%\end{thebibliography}


\chapter{Script system}

\noindent\textbf{Namespaces:} ScriptSystem

\noindent\textbf{Headers:} ScriptMgr.h, ScriptRegister.h, ScriptResource.h

\noindent\textbf{Source files:} ScriptMgr.cpp, ScriptRegister.cpp, ScriptResource.cpp

\noindent\textbf{Classes:} ScriptMgr, ScriptResource

\noindent\textbf{Libraries used:} AngelScript

\section{Purpose of script system}

\section{Introduction to used script language}
\label{script-language}

\section{Linking script engine to game engine}

% explain engine, module, source file, context, function

\subsection{Interface of script manager}

The script manager is represented by the class \emph{ScriptSystem::ScriptMgr} that encapsulates the script engine and manages an access to the script modules. This class is a singleton and it is created when the game starts. It initializes the AngelScript engine and registers all integral classes and functions (see \ref{sub:script-registered}) as well as all user-defined ones (see \ref{script-registering}).

The first thing a caller of a script function must do is to obtain a function ID. It can be get from the method \emph{ScriptMgr::GetFunctionID} that needs the name of the module where the function is and its declaration. For example if the name of function to be called is ``IncreaseArgument'' and it receives one integer argument and returns also an integer, the declaration of it will be ``int32 IncreaseArgument(int32)''. The method returns the integer that means the desired function ID (greater than or equal to zero) or the error code (less than zero) which means the function with this declaration could not be find in the mentioned module or this module does not exist or cannot be builded from sources (see log for further information). The function ID is valid all time the module exists in memory so it can be stored for later usage.

The function mentioned above calls \emph{ScriptMgr::GetModule} to obtain desired module. This method returns the module from a memory or loads its sources from disc and builds it if necessary. If it is inconvenient that the module is builded at the first call of its function, this method can be called before to get a confidence that the module is ready to use.

There are two ways of calling script functions. The first one is used when the script function should be called entire at once and the caller needs the return value of it. It can done with two methods. First the caller calls \emph{ScriptMgr::PrepareContext} which needs function ID and returns context prepared for passing the argument values. See the script documentation \cite{angelscript} or an example in \ref{script-component} for appropriate methods. Then the \emph{ScriptMgr::ExecuteContext} should be called with the prepared context as the first argument and a maximum time of executing script to prevent cycling as the second one. This method executes the script with given arguments and returns whether the execution was successful. If so it is possible to get return value of function with corresponding methods of context. In the end the context should be released to avoid memory leaks.

The second way of calling script function is used when it should run for a longer time with breaks caused by calling sleep and yield functions. In this case contexts are created within the context manager and they are executed by it subsequently when the \emph{ScriptMgr::ExecuteScripts} method is called which should be done in a game loop. Script functions are added to context manager by \emph{ScriptMgr::AddContextToManager} and \emph{Script\-Mgr\-::\-Add\-Context\-As\-Co\-Routine\-To\-Manager\-} which both needs a function ID as a parameter and returns context prepared for passing arguments. The first method adds the context as a new thread independent of others whereas the second one adds it as co-routine to the context passed by parameter which means when one of the co-routined contexts calls the yield function then the next context in queue will resume executing. The context added to the manager is released by itself when it finishes its execution.

As mentioned in the section \ref{script-language} it is possible to use a conditional compilation of scripts. The method \emph{ScriptMgr::DefineWord} adds a pre-processor define passed as the string argument. The last method of this class is \emph{ScriptMgr::ClearModules} that unloads all previously loaded and builded modules and abort all contexts in context manager. All function IDs and contexts got from methods of this class will be superseded so the caller should inform all objects that holds them about it. This method could be called when it is needed to free all memory used by modules but when it is better not to destroy the whole script engine.

\subsection{Binding entity system}
\label{script-entity}

The script system is binded to the entity system to provide an easy work with components, entities and theirs properties from scripts. The class \emph{EntityHandle} is registered to the script engine with a most of its methods but the work with entity properties differs from using them from a source code. There are registered common methods of this class for all defined property types (in \emph{PropertyTypes.h}) to get and set a simple property value, to get a non-constant or constant array and to call a property function with parameters. Methods for working with array property values (get and set a size of array, an index operator) and property function parameters (add a simple or array value as a property function parameter) are also registered. See table \ref{tab:script-properties} for method declarations.

\begin{table}[htbp]
	\centering
		{\scriptsize 
		\begin{tabular}{|p{0.38\hsize}|p{0.55\hsize}|}
			\hline
			An action & A declaration\\
			\hline
			\multicolumn{2}{|c|}{\emph{EntityHandle} methods}\\
			\hline
			Get a simple property value & \verb/type Get_type(string& property_name)/\\
			Set a simple property value & \verb/void Set_type(string& property_name, type value)/\\
			Get a non-constant array property value & \verb/array_type Get_array_type(string& property_name)/\\
			Get a constant array property value & \verb/const array_type Get_const_array_type/ \verb/  (string& property_name)/\\
			Call an entity function & \verb/void CallFunction(string& function_name,/ \verb/  PropertyFunctionParameters& parameters)/\\
			\hline
			\multicolumn{2}{|c|}{\emph{array\_type} methods}\\
			\hline
			Get a size of array & \verb/int32 GetSize() const/\\
			Set a size of non-constant array & \verb/void Resize(int32 size)/\\
			An index for a constant array & \verb/type operator[](int32 index) const/\\
			An index for a non-constant array & \verb/type& operator[](int32 index)/\\
			\hline
			\multicolumn{2}{|c|}{\emph{PropertyFunctionParameters} methods}\\
			\hline
			Add a simple property as a property function parameter & \verb/PropertyFunctionParameters operator<</ \verb/  (const type& value)/\\
			Add an array property as a property function parameter & \verb/PropertyFunctionParameters operator<</ \verb/  (const array_type& value)/\\
			\hline
		\end{tabular}
		}
	\caption[Working with entity properties]{Declarations of methods for working with entity properties where type means simple property type from \emph{PropertyTypes.h}}
	\label{tab:script-properties}
\end{table}

For example if an entity represented by its handle, that can be get from \verb/GetCurrentEntityHandle/ global function when script is handling a message to entity (see section \ref{script-component} for details), has an integer property \verb/Integer/ then the command to save it to an integer script variable will be following: \verb/int32 n = GetCurrentEntityHandle().Get_int32("Integer")/. If the property \verb/Integer/ does not exist in this entity or has an other type than \verb/int32/ the error message is written to the log and the default value (for \verb/int32/ 0) is returned.

For managing entities the \emph{EntityMgr} class is registered as no-handle object which means that only way to use it is calling methods on the return value from \verb/GetEntityMgr/ global function. For example if the script should destroy the entity represented by the variable \verb/handle/, the code will be following: \verb/GetEntityMgr().DestroyEntity(handle)/. For creation of entities the \emph{EntityDescription} class is registered, where it is possible to specify contained components, name, kind etc., as well as method \emph{CreateEntity} of \emph{EntityMgr} which has this class as parameter and returns handle of created entity.

\subsection{Registered classes and functions}
\label{sub:script-registered}

Beside the classes from the entity system mentioned in the section \ref{script-entity}, other base classes and functions are registered to the script engine. For storing a text data there are two classes. The first one is a standard C++ \emph{string} class registered with a most of its method, the second one is a class \emph{StringKey} which is an integer representation of text for faster comparing and assignment and which can be construct from \emph{string} and which can be cast to \emph{string} by the \emph{ToString} method.

For a thread and co-routine support there are registered three global functions which should be used when the context is executed by the context manager. The function \verb/void sleep(uint32 time)/ suspends the script execution for at least \verb/time/ milliseconds so contexts in the other threads can be executed. The function \verb/void createCoRoutine(const string& declaration)/ creates a co-routine to the current context that calls the script function in the same module without parameters with the declaration equal to the parameter of this function. The function \verb/void yield()/ suspends the current script execution and executes the next co-routine from the same thread. For example calling the last two function subsequently means that the co-routine is created and immediately executed while the current execution is suspended waiting for the co-routine function calling \verb/yield()/.

The last registered global function is \verb/void Log(string& message)/ that prints the string in parameter to the log so it is useful for debugging scripts.

\section{Using and extending script system}

% situations for registering new classes

\subsection{Registering new classes and functions}
\label{script-registering}

If it is necessary to register a new class with its methods or a global function, the registration function should be implemented in the \emph{ScriptRegister.cpp} file and called by the global function \emph{RegisterAllAdditions} with the pointer to script engine as a parameter because a registration is done by calling methods of it.

The registration function should first declare the integer variable \verb/int32 r/ to which should be assign all return values from calling registration methods and after each calling the \verb/OC_SCRIPT_ASSERT()/ should be used to check if the registration is successful (it checks the variable \verb/r/ to be positive). First it is necessary to register the whole class which is done by the \emph{RegisterObjectType} method. The first parameter is a name of class in a script, the second is a size of class (use \verb/sizeof(class)/ for register a class as value, \verb/0/ otherwise) and the last parameter is flag. See table \ref{tab:script-objecttype} or documentation of AngelScript \cite{angelscript} for some flag combination.

\begin{table}[htbp]
	\centering
		{\scriptsize
		\begin{tabular}{|p{0.43\hsize}|p{0.50\hsize}|}
		\hline
		A class type & Flags\\
		\hline
		A primitive value type without any special management & \verb/asOBJ_VALUE | asOBJ_POD | asOBJ_APP_CLASS_CA/\\
		A value type needed to be properly initialized and uninitialized & \verb/asOBJ_VALUE | asOBJ_APP_CLASS_CA/\\
		A basic reference type & \verb/asOBJ_REF/\\
		A single-reference type (singleton) & \verb/asOBJ_REF | asOBJ_NOHANDLE/\\
		\hline
		\end{tabular}
		}
	\caption{Flags in register function method}
	\label{tab:script-objecttype}
\end{table}

After the registration the whole class it is possible to register methods of it with the \emph{RegisterObjectMethod} method. The first parameter is the name of the registered class, the second is the declaration of method in a script, the third is a pointer to the C++ function that should be called and the last is the calling convention. The pointer to C++ function should be get from one of the four macros described in the table \ref{tab:script-macros} (\verb/PR/ variants must be used when functions or methods are overloaded) and the possible calling conventions are listed in the table \ref{tab:script-conventions}. For example the registering of method \emph{bool EntityHandle::IsValid() const} as a method of the \verb/EntityHandle/ script class the code will be following:

{\footnotesize\begin{verbatim}
r = engine->RegisterObjectMethod("EntityHandle", "bool IsValid() const",
  asMETHOD(EntityHandle, IsValid), asCALL_THISCALL); OC_SCRIPT_ASSERT();
\end{verbatim}}

\begin{table}[htbp]
	\centering
		{\scriptsize
		\begin{tabular}{|p{0.55\hsize}|p{0.38\hsize}|}
		\hline
		A macro declaration & An example\\
		\hline
		\verb/asFUNCTION(global_function_name)/ & \verb/asFUNCTION(Add)/\\
		\verb/asFUNCTIONPR(global_function_name, (parameters),/ \verb/  return_type)/ & \verb/asFUNCTIONPR(Add, (int), void)/\\
		\verb/asMETHOD(class_name, method_name)/ & \verb/asMETHOD(Object, Add)/\\
		\verb/asMETHODPR(class_name, method_name, (parameters),/ \verb/  return_type)/ & \verb/asMETHOD(Object, Add, (int), void)/\\
		\hline
		\end{tabular}
		}
	\caption{Used macros for a function and method registration}
	\label{tab:script-macros}
\end{table}

\begin{table}[htbp]
	\centering
		{\scriptsize
		\begin{tabular}{|p{0.24\hsize}|p{0.69\hsize}|}
		\hline
		A call convention & An use case\\
		\hline
		\verb/asCALL_CDECL/ & \verb/A cdecl function (for global functions)/\\
		\verb/asCALL_STDCALL/ & \verb/A stdcall function (for global functions)/\\
		\verb/asCALL_THISCALL / & \verb/A thiscall class method (for class methods)/\\
		\verb/asCALL_CDECL_OBJLAST / & \verb/A cdecl function with the object pointer as the last parameter/\\
		\verb/asCALL_CDECL_OBJFIRST / & \verb/A cdecl function with the object pointer as the first parameter/\\
		\hline
		\end{tabular}
		}
	\caption{Used calling conventions for a function and method registration}
	\label{tab:script-conventions}
\end{table}

Operators are registered as common methods but they have a special script declaration name such as \verb/opEquals/ or \verb/opAssign/. See AngelScript documentation \cite{angelscript} for further operator names.

If the class has some special constructors or destructor, they need to be registered by the \emph{RegisterObjectBehaviour} method which has same parameters as the \emph{RegisterObjectMethod} except an inserted second parameter that means which behavior is registered (see table \ref{tab:script-behaviors} for possible values). They must be registered by proxy functions that take a pointer to an object as the last argument (so the \verb/asFUNCTION/ macro and the \verb/asCALL_CDECL_OBJLAST/ calling convention are used) and fill it with a constructed object or call a correct destructor on it.

If the class is registered as a basic reference type, it needs to have a reference counter, a factory function and add-reference and release methods. These methods are registered by \emph{RegisterObjectBehaviour} method as well as constructor and destructor. The single-reference type does not need these methods but it cannot be passed as function parameter or stored to variable.

\begin{table}[htbp]
	\centering
		{\scriptsize
		\begin{tabular}{|p{0.63\hsize}|p{0.30\hsize}|}
		\hline
		A C++ function declaration & A behavior\\
		\hline
		\verb/void ObjectDefaultConstructor(Object* self)/ & \verb/asBEHAVE_CONSTRUCT/\\
		\verb/void ObjectCopyConstructor(Object& other, Object* self)/ & \verb/asBEHAVE_CONSTRUCT/\\
		\verb/void ObjectDestructor(Object* self)/ & \verb/asBEHAVE_DESTRUCT/\\
		\verb/Object *ObjectFactory()/ & \verb/asBEHAVE_FACTORY/\\
		\verb/void Object::Addref()/ & \verb/asBEHAVE_ADDREF/\\
		\verb/void Object::Release()/ & \verb/asBEHAVE_RELEASE/\\
		\hline
		\end{tabular}
		}
	\caption{Possible behaviors for the \emph{RegisterObjectBehaviour} method}
	\label{tab:script-behaviors}
\end{table}

It is also possible to register class properties by the \emph{RegisterObjectProperty} method but more portable is to register theirs getters and setters as class methods with strict declarations (\verb/type get_propname() const/ and \verb/void set_propname(type)/) so they can be used almost as registered directly (\verb/object.propname/).

Global functions are registered by the \emph{RegisterGlobalFunction} method that has almost same parameters as the \emph{RegisterObjectMethod} (does not have the first), global properties by \emph{RegisterGlobalProperty} method. There are also methods for registering enumerations (\emph{RegisterEnum}, \emph{RegisterEnumValue}) and typedefs for primitive types (\emph{RegisterTypedef}) that are easy to use. Further information about registering types can be found at the AngelScript documentation \cite{angelscript}, examples of using are in the \emph{ScriptRegister.cpp} file.

\subsection{Sample component}
\label{script-component}

\subsection{Tip for using context manager}

\section{Possible extension}

\section{Glossary}

\begin{thebibliography}{9}                                                                                                
\bibitem {angelscript}AngelScript documentation, file /AngelScript/index.html

\end{thebibliography}

\chapter{String system}
\label{chap:string}

\begin{description}
  \item[Namespaces:] StringSystem
  \item[Headers:] FormatText.h, StringMgr.h, TextData.h, TextResource.h
  \item[Source files:] FormatText.cpp, StringMgr.cpp, TextResource.cpp
  \item[Classes:] FormatText, StringMgr, TextResource
  \item[Libraries used:] CEGUI
\end{description}

\section{Purpose of the string system}

The string system manages all texts that are visible to the user except internal log messages. It supports a localization to various languages and their country-specific dialects and a switching among them on fly.

In the following sections the required format of text files and the directory layout will be described as well as switching the languages and loading and formating the desired localized text. In the last section there is a small glossary of used terms.

\section{Format of text files}

One text file consists of lines of text items and comments. The pattern of a text item is \verb/key=Value/, where \verb/key/ is an ID that is used for indexing a text represented by \verb/Value/ from the application. The first text item in a file defines a group for the following text items if its \verb/key/ is \verb/group/, otherwise the group is set to the default one. An ID must be unique within a group, must not begin with the \verb/#/ character and contain the \verb/=/ character and the same ID in the same group should represented the same text in each language.

A \verb/Value/ part can contain \verb/%x/ character sequences where \verb/x/ is a number from 1 to 9. When a text with these sequences is loaded it is possible to replace them with another text in order according to the specific number. See the section \ref{sec:text-format} for more information.

A comment is a text that begins with the \verb/#/ character and it is ignored by the application. Every text item and each comment must be on it's own line without any leading white characters. Here is an example of a correct text file:

\begin{verbatim}
# the name of the group to which the following texts belong
group=ExampleGroup
# comment to the text item below
example_key=Example text.
another_key=Another text.
\end{verbatim}

The encoding of text files must be ASCII or UTF8. It is possible to end lines in Windows (\verb/\r\n/) or UNIX (\verb/\n/) style.

\section{Directory layout}
\label{sec:dir-layout}

The directory in which the text files are stored must have a specific layout. In the root directory there should be files with default language texts. If any ID of a text item of a specific group is not contained in a chosen language subdirectory, the text item of a default language will be used. In this directory there should be also subdirectories representing language specific texts. It is recommended that their names correspond with ISO 639-1 codes \cite{ISO-639-1} of their language (i.e. \verb/en/ for English, \verb/fr/ for French etc.). These subdirectories should contain the same files as the root directory which should consist of the language specific text items that will be preferably used if the corresponding language is chosen.

If any text item varies among various countries that use a common language it should be placed to the same file in subdirectories of a language directory. It is recommended that their names correspond with ISO 3166-1 alpha-2 codes \cite{ISO-3166-1} of a country they represent (i.e. \verb/US/ for United States, \verb/GB/ for United Kingdom etc.). Here is an example of a directory layout:

\begin{verbatim}
-en
 -GB
  -textfile.str (1)
 -US
  -textfile.str (2)
 -textfile.str  (3)
-fr
 -textfile.str  (4)
-textfile.str   (5)
\end{verbatim}

Assume that the files from the above example have a following content:

\begin{verbatim}
(1)
group=Group
country_specific=Country
(3)
group=Group
language_specific=Language
country_specific=Language
(5)
group=Group
default_string=Default
language_specific=Default
country_specific=Default
\end{verbatim}

The texts given from invoking the following IDs of group \verb/Group/ for specific languages are shown in the table \ref{tab:dir-layout}.

\begin{table}[htbp]
  \begin{center}
	\begin{tabular}{|l|l|}
	\hline
	Text ID & Text value\\
	\hline
	\multicolumn{2}{|c|}{\emph{Language: en-GB}}\\
	\hline
	default\_string & Default\\
	language\_specific & Language\\
	country\_specific & Country\\
	\hline
	\multicolumn{2}{|c|}{\emph{Language: en}}\\
	\hline
	default\_string & Default\\
	language\_specific & Language\\
	country\_specific & Language\\
	\hline
	\multicolumn{2}{|c|}{\emph{Language: default}}\\
	\hline
	default\_string & Default\\
	language\_specific & Default\\
	country\_specific & Default\\
	\hline
	\end{tabular}
	\caption{Texts given from IDs with various language settings}
	\label{tab:dir-layout}
	\end{center}
\end{table}

\section{Interface of the string manager}

The string manager is represented by the class \emph{StringSystem::StringMgr} that provides all necessary services. There are two instances of this class when the application runs. The first one manages system texts (i.e. labels in editor, common error messages etc.), the second one provides an access to project specific texts, so they differs only in a path to the root directory described in the section \ref{sec:dir-layout}. Both are initialized when the application starts and destroyed on an application shutdown. The class provides static methods returning these instances or it is possible to use defined macros.

The first method that is necessary to call before using the others is the \emph{StringMgr::LoadLanguagePack} that expects the language and the country code (both can be omitted) which meaning is widely explained in the section \ref{sec:dir-layout}. It removes old text items and loads new ones according to a specific language setting to the memory and divides them to the desired groups. This method can be called during the whole execution of application but the other systems must refresh their data themselves.

There are several methods in the class for getting the text data according to its group and ID which differs in returning a pointer to the data or the data itself and in specifying or omitting the group name (using the default one instead). If the text data of a specified group and ID does not exist, an empty string is returned and an error message is written to the log.

\section{Using a variable text}
\label{sec:text-format}

If the loaded text data contains character sequences in a form of \verb/%x/ (\verb/x/ is a number from 1 to 9) it is possible to replace them with another text using the class \emph{StringSystem::TextFormat}. Just construct the instance of this class with the loaded text as a parameter, then use a sequence of \verb/<</ operators to replace all well formated character sequences and store it to another text data variable.

The \verb/<</ operator finds the described character sequence with the minimum number and replaces it with the text provided as the parameter. If no such sequence is found then it inserts the provided text at the end of the loaded text. For example if the code

\begin{verbatim}
StringSystem::TextData loaded = "The %2, the %3 and the %1.";
StringSystem::TextData result = StringSystem::TextFormat(loaded)
    << "third" << "first" << "second";
\end{verbatim}

\noindent is called then in the \verb/result/ variable there will be the following text:

\begin{verbatim}
The first, the second and the third.
\end{verbatim}


\section{Glossary}
This is a glossary of the most used terms in the previous sections:

\begin{description}
  \item[Key] -- an ID that is used for indexing a text data, must be unique within a group
  \item[Text item] -- a pair of a key and a text data that is contained it a text file
  \item[Group] -- a set of text items that can be indexed from application, one file contains text items from one group, text items from one group can be contained in several files
  \item[Language code] -- two-character acronym representing a world language according to ISO 639-1 \cite{ISO-639-1}.
  \item[Country code] -- two-character acronym representing a country according to ISO 3166-1 alpha-2 \cite{ISO-3166-1}.
  \item[ASCII] -- an encoding of text that uses only numbers, characters of English alphabet and a few symbols and non-printing control characters
  \item[UTF8] -- a backward compatible ASCII encoding extension that is able to represented any character in the Unicode standard in 1 to 4 bytes
\end{description}

\chapter{Platform setup}

\begin{description}
  \item[Namespaces:] none
  \item[Headers:] BasicTypes.h, ComplexTypes.h, Containers.h, Memory\_post.h, Me\-mo\-ry\-\_pre.h, Platform.h, Settings.h
  \item[Source files:] ComplexTypes.cpp
  \item[Classes:] none
  \item[Libraries used:] none
\end{description}


\section{Introduction}
To be able to build the engine for different platforms it was decided to concentrate all platform specific settings into one place. These include basic type definitions, macros, standard header file includes and containers. All of these can be found in the header files in the \verb'Setup' directory under the source tree.


\section{Specific header files}
The main header file included in most compilation units is \verb'Settings.h'. It contains basic settings for the platform (macros controlling the behaviour of libraries for example), but more importantly it aggregates other setup headers together.

In \verb'Platform.h' you can find macros defining the currently used platform. These macros are then used in other parts of the project to branch the code for specific platforms in compile time. In \verb'BasicTypes.h' there are definitions of simple types used in the whole project (integer, float, etc.). In \verb'Containers.h' there are definitions of the STL-like containers. And \verb'ComplexTypes.h' contains definitions of other complex STL-like data structures.

\verb'Memory.h' defines the memory allocation method used by the project and includes their implementation from the memory subsystem (the \verb'Memory' directory).


%\section{Glossary}
%This is a glossary of the most used terms in the previous sections:
%
%\begin{description}
%  \item[Resource] -- a unit of data the game will be working with as a whole. The data is usually stored in an external device.
%\end{description}


%\begin{thebibliography}{9}                                                                                                
%\bibitem {angelscript}AngelScript documentation, file /AngelScript/index.html
%\end{thebibliography}


\chapter{Utils}
\label{chap:utils}

\noindent\textbf{Sources:} the Utils directory


\section{Introduction}
During the development of a game you usually need several helper classes and methods such as those for math or containers. This kind of stuff doesn't fit anywhere because it's too general. Placing it in a specific subsystem would make it unusable anywhere else. So it's best to aggregate this in one place.



\section{Categories}
There are several categories of utilities: containers (ie. tree), math functions (ie. hash) and design patterns (ie. singleton). The utilities are usually simple enough to understand, so it's best to read the code and/or the doxygen documentation. However, a few of them are a bit more complicated. That's why they're described in the following sections.


\subsection{RTTI}
RTTI is a shortage for \emph{Run-Time Type Information}. It's a mechanism which allows an instance of a class to know what its class is, the name of the class and other attributes. The instance is also able to create new instances of the same class, cloning itself.

RTTI is implemented using C++ templates. To make a use of it you must derive your class from \verb'RTTIGlue' which takes two template arguments. The first one is your class and the second one is its predecessor in the RTTI hierarchy. If there is none or you don't want to specify that inherit \verb'RTTIBaseClass' instead of \verb'RTTIGlue'. The mechanism will automatically call the static \verb'RegisterReflection()' function of your class if defined. The function is called only once during the startup. Here you may adjust attributes of the class or add new ones.


\subsection{Properties}
\emph{Properties} are special \emph{RTTI} attributes of classes which allow accessing data of their instances using string names. This greatly helps encapsulating classes and provides an uniform data access interface.

To make use of \emph{Properties} your class must be already using \emph{RTTI}. Create a getter and setter method for each of your data you want to provide as properties. Define the static \verb'RegisterReflection()' function (see the RTTI section for details) and inside its body call \verb'RegisterProperty()' providing pointers to the data getters and setters. For details about this function see the doxygen docs. To access the data of an instance use \verb'GetRTTI()->GetProperty()' on the instance. It will give you \verb'PropertyHolder' on which you can finally call the templatized \verb'GetValue()' or \verb'SetValue()' methods.

You can also set a custom function as a special property. The function must take \verb'PropertyFunctionParameters' as a single parameter and return \verb'void'. You can register it using \verb'RegisterFunction()'. The function can be then used in a similar way as the common properties, but you use \verb'CallFunction()' instead of \verb'SetValue()'.


%\section{Glossary}
%This is a glossary of the most used terms in the previous sections:
%
%\begin{description}
%  \item[Resource] -- a unit of data the game will be working with as a whole. The data is usually stored in an external device.
%\end{description}


%\begin{thebibliography}{9}                                                                                                
%\bibitem {angelscript}AngelScript documentation, file /AngelScript/index.html
%\end{thebibliography}


%%%%%%%%%%%%%%%%%%%%%%%%%%%%%%%%%%%%%%%%%%%%%%%%%%%%%%%%%%%%%
%% BIBLIOGRAPHY AND OTHER LISTS
%%%%%%%%%%%%%%%%%%%%%%%%%%%%%%%%%%%%%%%%%%%%%%%%%%%%%%%%%%%%%
%% A small distance to the other stuff in the table of contents (toc)
\addtocontents{toc}{\protect\vspace*{\baselineskip}}

%% The Bibliography
%% ==> You need a file 'literature.bib' for this.
%% ==> You need to run BibTeX for this (Project | Properties... | Uses BibTeX)

\begin{thebibliography}{9}
\addcontentsline{toc}{chapter}{Bibliography} %'Bibliography' into toc
\bibitem {angelscript}AngelScript documentation, file /AngelScript/index.html
\bibitem {ISO-639-1}ISO 639-1 -- http://en.wikipedia.org/wiki/List\_of\_ISO\_639-1\_codes
\bibitem {ISO-3166-1}ISO 3166-1 alpha-2 -- http://en.wikipedia.org/wiki/ISO\_3166-1\_alpha-2
\end{thebibliography}


%\nocite{*} %Even non-cited BibTeX-Entries will be shown.
%\bibliographystyle{alpha} %Style of Bibliography: plain / apalike / amsalpha / ...
%\bibliography{literature} %You need a file 'literature.bib' for this.

%% The List of Figures
\clearpage
\addcontentsline{toc}{chapter}{List of Figures}
\listoffigures

%% The List of Tables
\clearpage
\addcontentsline{toc}{chapter}{List of Tables}
\listoftables

\appendix

\chapter{Script system registered object}

\section{Purpose of this document}

This document is an enumeration of classes, global functions etc. that are registered to the script engine so they can be used from scripts. It is divided to the section about objects from the game engine and to the section about additional user-defined objects.

\section{Integral registered objects}

This section is about registered objects from the game engine so it should not be edited until the game engine is changed.

\subsection{Classes}

\subsubsection{StringKey}

This class serves as a key into maps and other structures where we want to index data using strings, but we need high speed as well. The string value is hashed and the result is then used as a decimal representation of the string. This class is registered as a value type.

\begin{titled-itemize}{Constructors}
	\item StringKey() -- default constructor
	\item StringKey(const StringKey \&in) -- copy constructor
	\item StringKey(const string \&in) -- constructs the key from a standard string
\end{titled-itemize}

\begin{titled-itemize}{Operators and methods}
  \item bool operator=(const StringKey \&in) const -- assignment operator
  \item StringKey\& operator==(const StringKey \&in) -- equality operator
  \item string ToString() const -- converts the key to a string
\end{titled-itemize}

\subsubsection{array\_T}

This is a group of classes parametrized by a type T from PropertyTypes.h that serve as an array of values of a type T corresponding to properties of the Array$<$T$>$* type. For example an array of 32-bit signed integer will be declared as \verb/array_int32/. These classes are not compatible with an script array defined as for example \verb/int32[]/ but they have similar methods. This class is registered as a value type.

\begin{titled-itemize}{Constructors}
  \item array\_T() -- default constructor that should not be used, instances of these classes are got from Get\_array\_T and Get\_const\_array\_T methods of EntityHandle class
\end{titled-itemize}

\begin{titled-itemize}{Operators and methods}
  \item T\& operator[](int32) -- write accessor to an array item
  \item T operator[](int32) const -- read accessor to an array item
  \item int32 GetSize() const -- returns a size of the array
  \item void Resize(int32) -- resize an array to a new size
\end{titled-itemize}

\subsubsection{PropertyFunctionParameters}

This class represents generic parameters passed to a function accessed via the properties system. Thanks to the $<<$ operators the function parameter can be passed as \verb/PropertyFunctionParameters()/ \verb/<< param1 << param2 .../. This class is registered as a value type.

\begin{titled-itemize}{Constructors}
  \item PropertyFunctionParameters() -- constructs new empty parameters
\end{titled-itemize}

\begin{titled-itemize}{Operators and methods}
  \item PropertyFunctionParameters\& operator=(const PropertyFunctionParameters \&in) -- assignment operator
  \item bool operator==(const PropertyFunctionParameters \&in) const -- e\-qua\-li\-ty operator
  \item PropertyFunctionParameters operator$<<$(const T \&in) const -- add parameter of type T from PropertyTypes.h
  \item PropertyFunctionParameters operator$<<$(const array\_T \&in) const -- add parameter of type array\_T
\end{titled-itemize}

\subsubsection{EntityHandle}

This class represents one unique entity in the entity system. This class is registered as a value type.

\begin{titled-itemize}{Constructors}
  \item EntityHandle() -- default constructor will initialize the handle to an invalid state
  \item EntityHandle(const EntityHandle \&in) -- only the copy constructor is enabled, new entities should be added only by the EntityMgr
\end{titled-itemize}

\begin{titled-itemize}{Operators and methods}
  \item EntityHandle\& operator=(const EntityHandle \&in) -- assignment operator
  \item bool operator==(const EntityHandle \&in) -- equality operator
  \item bool IsValid() const -- returns true if this handle is valid (not null)
  \item bool Exists() const -- returns true if this entity still exists in the system
  \item EntityID GetID() const -- returns the internal ID of this entity
  \item string GetName() const -- returns the name of this entity
  \item EntityTag GetTag() const -- returns the tag of this entity
  \item void SetTag(EntityTag) -- sets the tag of this entity
  \item eEntityMessageResult PostMessage(const eEntityMessageType, PropertyFunctionParameters = PropertyFunctionParameters()) -- sends a message to this entity (message type, parameters)
  \item void CallFunction(string \&in, PropertyFunctionParameters \&in) -- calls a function on an entity of a specific name and with specific parameters
  \item T Get\_T(string \&in) -- gets a value of an entity property of a specific name and type T from PropertyTypes.h
  \item void Set\_T(string \&in, T) -- sets a value of an entity property of a specific name and type T from PropertyTypes.h
  \item array\_T Get\_array\_T(string \&in) -- gets a non-constant value of an entity property of a specific name and type of array\_T
  \item const array\_T Get\_const\_array\_T(string \&in) -- gets a constant value of an entity property of a specific name and type of array\_T
  \item bool RegisterDynamicProperty\_T(const string \&in, const PropertyAccessFlags, const string \&in) -- registers an entity dynamic property of a specific name, access and comment
  \item bool UnregisterDynamicProperty(const string \&in) const -- unregisters an entity dynamic property of a specific name
\end{titled-itemize}

\subsubsection{Vector2}

This class represents 2D vector and it is registered as a value type.

\begin{titled-itemize}{Properties}
  \item float32 x -- x coordinate
  \item float32 y -- y coordinate
\end{titled-itemize}

\begin{titled-itemize}{Constructors}
  \item Vector2() -- default constructor (x = y = 0)
  \item Vector2(const Vector2 \&in) -- copy constructor
  \item Vector2(float32 x, float32 y)  -- constructor using coordinates
\end{titled-itemize}

\begin{titled-itemize}{Operators and methods}
  \item void operator+=(const Vector2 \&in) -- add a vector to this vector
  \item void operator-=(const Vector2 \&in) -- subtract a vector from this vector
  \item void operator*=(float32) -- multiply this vector by a scalar
  \item Vector2 operator-() const -- negate this vector
  \item bool operator==(const Vector2 \&in) const -- equality operator
  \item Vector2 operator+(const Vector2 \&in) const -- returns a sum of this and argument vector
  \item Vector2 operator-(const Vector2 \&in) const -- returns a difference of this and argument vector
  \item Vector2 operator*(float32) const -- returns a product of a scalar and this vector
  \item float32 Length() const -- get the length of this vector (the norm)
  \item float32 LengthSquared() const -- get the length squared
  \item void Set(float32, float32) -- set this vector to some specified coordinates
  \item void SetZero() -- set this vector to all zeros
  \item float32 Normalize() -- converts this vector into a unit vector and returns the length
  \item bool IsValid() const -- returns whether this vector contain finite coordinates
  \item float32 Dot(const Vector2 \&in) const -- returns a scalar product of this and argument vector
\end{titled-itemize}

\subsubsection{EntityPicker}

This class allows the game to pick an entity based on provided input data. This class is registered as a value type.

\begin{titled-itemize}{Constructors}
  \item EntityPicker(const Vector2 \&in, const int32, const int32) -- creates a picker for selecting an entity under the current mouse cursor
\end{titled-itemize}

\begin{titled-itemize}{Operators and methods}
  \item EntityHandle PickSingleEntity() -- runs the picking query, the result is returned directly
  \item void PickMultipleEntities(EntityHandle[] \&out, const Vector2 \&in, const float32) -- runs the picking query, the result is filled into the given array, the query is defined by a rectangle between the last cursor position and the given cursor position, the rectangle is rotated by the given angle
\end{titled-itemize}

\subsubsection{EntityDescription}

This class contains all info needed to create one instance of an entity. It is basically a collection of component descriptions. This class is registered as a value type.

\begin{titled-itemize}{Constructors}
  \item EntityDescription() -- default constructor
\end{titled-itemize}

\begin{titled-itemize}{Operators and methods}
  \item void Reset() -- clears everything, call this before each subsequent filling of the description
  \item void AddComponent(const eComponentType) -- adds a new component specified by its type
  \item void SetName(const string \&in) -- sets a custom name for this entity
  \item void SetKind(const eEntityDescriptionKind) -- sets this entity to be an ordinary entity or a prototype
  \item void SetPrototype(const EntityHandle) -- sets the prototype the entity is to be linked to
  \item void SetPrototype(const EntityID) -- sets the prototype the entity is to be linked to
  \item void SetDesiredID(const EntityID) -- sets a desired ID for this entity, it does not have to be used by EntityMgr
\end{titled-itemize}

\subsubsection{EntityMgr}

This class manages all game entities. It is registered as a no-handle reference so the only way to use it is to get reference from a global property.

\begin{titled-itemize}{Operators and methods}
  \item EntityHandle CreateEntity(EntityDescription \&in) -- creates a new entity accordingly to its description and returns its handle
  \item EntityHandle InstantiatePrototype(const EntityHandle, const string \&in) -- creates a new entity from a prototype with a specific name
  \item EntityHandle DuplicateEntity(const EntityHandle, const string \&in) -- duplicates an entity with a specific new name
  \item void DestroyEntity(const EntityHandle) -- destroys a specified entity if it exists
  \item bool EntityExists(const EntityHandle) const -- returns true if the entity exists
  \item EntityHandle FindFirstEntity(const string \&in) -- returns EntityHandle to the first entity of a specified name
  \item EntityHandle GetEntity(EntityID) const -- return EntityHandle of the entity with specified ID
  \item bool IsEntityInited(const EntityHandle) const -- returns true if the entity was fully initialized
  \item bool IsEntityPrototype(const EntityHandle) const -- returns true if the entity is a prototype
  \item void LinkEntityToPrototype(const EntityHandle, const EntityHandle) -- assigns the given entity to the prototype
  \item void UnlinkEntityFromPrototype(const EntityHandle) -- destroys the link between the component and its prototype
  \item bool IsPrototypePropertyShared(const EntityHandle, const String\-Key) const -- returns true if the property of the prototype is marked as shared (and thus propagated to instances)
  \item void SetPrototypePropertyShared(const EntityHandle, const String\-Key) -- marks the property as shared among instances of the prototype
  \item void SetPrototypePropertyNonShared(const EntityHandle, const String\-Key) -- marks the property as non shared among instances of the prototype
  \item void UpdatePrototypeInstances(const EntityHandle) -- propagates the current state of properties of the prototype to its instances
  \item bool HasEntityProperty(const EntityHandle, const StringKey, const PropertyAccessFlags) const -- returns true if the entity has the given property
  \item bool HasEntityComponentProperty(const EntityHandle, const ComponentID, const StringKey, const PropertyAccessFlags) const -- returns true if the component of the entity has the given property
  \item void BroadcastMessage(const eEntityMessageType, PropertyFunctionParameters = PropertyFunctionParameters()) -- sends a message to all entities
  \item bool HasEntityComponentOfType(const EntityHandle, const eComponentType) -- returns true if the given entity has a component of the given type
  \item int32 GetNumberOfEntityComponents(const EntityHandle) const -- returns the number of components attached to the entity
  \item ComponentID AddComponentToEntity(const EntityHandle, const e\-Com\-po\-nent\-Ty\-pe) -- adds a component of the specified type to the entity, returns an ID of the new component
  \item void DestroyEntityComponent(const EntityHandle, const Com\-po\-nent\-ID) -- destroys a component of the entity
\end{titled-itemize}

\subsubsection{MouseState}

This structure contains the state of the mouse device at a specific point of time. This class is registered as a simple value type.

\begin{titled-itemize}{Properties}
  \item int32 x -- x coordinate
  \item int32 y -- y coordinate
  \item int32 wheel -- wheel position
  \item uint8 buttons -- pressed buttons
\end{titled-itemize}

\subsubsection{InputMgr}

This class manages the game input. It is registered as a no-handle reference so the only way to use it is to get reference from a global property.

\begin{titled-itemize}{Operators and methods}
  \item void CaptureInput() -- updates the state of the manager and processes all events
  \item bool IsKeyDown(const eKeyCode) const -- returns true if a specified key is down
  \item bool IsMouseButtonPressed(const eMouseButton) const -- returns true if a specified button of the mouse is pressed
  \item MouseState\& GetMouseState() const -- returns the current state of the mouse
\end{titled-itemize}

\subsubsection{Project}

This class represents the current project. It is registered as a no-handle reference so the only way to use it is to get reference from a global property.

\begin{titled-itemize}{Operators and methods}
  \item bool OpenScene(const string \&in) -- opens the scene with given filename
  \item bool OpenSceneAtIndex(int32) -- opens the scene at the given index in the scene list
  \item uint32 GetSceneCount() const -- returns the number of scenes
  \item int32 GetSceneIndex(const string \&in) const -- returns the index of the scene with a specific name, -1 if does not exist
  \item string GetOpenedSceneName() const -- returns name of the opened scene, or empty string if no scene is opened
  \item string GetSceneName(int32) const -- returns name of the scene at the given index in the scene list
\end{titled-itemize}

\subsubsection{Game}

This class represents the current game. It is registered as a no-handle reference so the only way to use it is to get reference from a global property.

\begin{titled-itemize}{Operators and methods}
  \item void ClearDynamicProperties() -- clears the dynamic property list
  \item bool HasDynamicProperty(const string \&in) const -- returns whether the specified dynamic property exists
  \item bool DeleteDynamicProperty(const string \&in) -- deletes the specified dynamic property
  \item bool LoadFromFile(const string \&in) -- loads the game from the specified file
  \item void PauseAction() -- pauses the game action until resumed again
  \item void ResumeAction() -- resumes the game action if paused
  \item bool SaveToFile(const string \&in) -- saves the game to the specified file
  \item void Quit() -- quits the game
  \item uint64 GetTime() -- returns the current game time in miliseconds
  \item bool GetFullscreen() -- returns whether the game is in the fullscreen mode
  \item void SetFullscreen(const bool) -- sets whether the game is in the fullscreen mode
  \item T Get\_T(string \&in) -- gets a value of a dynamic property of a specific name and type T from PropertyTypes.h
  \item void Set\_T(string \&in, T) -- sets a value of a dynamic property of a specific name and type T from PropertyTypes.h
\end{titled-itemize}

\subsubsection{CEGUIString}

This class represents string in the CEGUI library. It is registered as a value type and it is implicitly cast from/to a common string so no other methods are needed.

\subsubsection{Window}

This class is the base class for all GUI elements. It is registered as a basic reference and its handle can be cast to all of its desceants.

\begin{titled-itemize}{Operators and methods}
  \item const CEGUIString\& GetName() const -- returns the name of this window
  \item const CEGUIString\& GetType() const -- returns the type name for this window
  \item bool IsDisabled() const -- returns whether the window is currently disabled (does not inherit state from ancestor windows)
  \item bool IsDisabled(bool) const -- returns whether the window is currently disabled (specify whether to inherit state from ancestor windows)
  \item bool IsVisible() const -- returns true if the window is currently visible (does not inherit state from ancestor windows)
  \item bool IsVisible(bool) const -- returns true if the window is currently visible (specify whether to inherit state from ancestor windows)
  \item bool IsActive() const -- returns true if this is the active window (may receive user inputs)
  \item const CEGUIString\& GetText() const -- returns the current text for the window
  \item bool InheritsAlpha() const -- returns true if the window inherits alpha from its parent(s)
  \item float32 GetAlpha() const -- returns the current alpha value set for this window (between 0.0 and 1.0)
  \item float32 GetEffectiveAlpha() const -- returns the effective alpha value that will be used when rendering this
 window
  \item Window\@ GetParent() const -- returns the parent of this window
  \item const CEGUIString\& GetTooltipText() const -- returns the current tooltip text set for this window
  \item bool InheritsTooltipText() const -- returns whether this window inherits tooltip text from its parent when
 its own tooltip text is not set
  \item void SetEnabled(bool) -- sets whether this window is enabled or disabled
  \item void SetVisible(bool) -- sets whether this window is visible or hidden
  \item void Activate() -- activates the window giving it input focus and bringing it to the top of
 all windows
  \item void Deactivate() -- deactivates the window
  \item void SetText(const CEGUIString\& in) -- sets the current text string for the window
  \item void SetAlpha(float32) -- sets the current alpha value for this window
  \item void SetInheritsAlpha(bool) -- sets whether this window will inherit alpha from its parent windows
  \item void SetTooltipText(const CEGUIString\& in) -- sets the tooltip text for this window
  \item void SetInheritsTooltipText(bool) -- sets whether this window inherits tooltip text from its parent when its
 own tooltip text is not set
\end{titled-itemize}

\subsubsection{ButtonBase}

This class is the base class for all button GUI elements. It is registered as a basic reference and its handle can be cast to all of its desceants and the Window class. Beside these methods it has all methods of the Window class.

\begin{titled-itemize}{Operators and methods}
  \item bool IsHovering() const -- returns true if user is hovering over this widget
  \item bool IsPushed() const -- returns true if the button widget is in the pushed state
\end{titled-itemize}

\subsubsection{Checkbox}

This class representes the checkbox GUI element. It is registered as a basic reference and its handle can be cast to the Window and ButtonBase classes. Beside these methods it has all methods of the Window and ButtonBase classes.

\begin{titled-itemize}{Operators and methods}
  \item bool IsSelected() const -- returns true if the checkbox is selected (has the checkmark)
  \item void SetSelected(bool) -- sets whether the checkbox is selected or not
\end{titled-itemize}

\subsubsection{PushButton}

This class representes the push button GUI element. It is registered as a basic reference and its handle can be cast to the Window and ButtonBase classes. It has all methods of the Window and ButtonBase classes and none more.

\subsubsection{RadioButton}

This class representes the radio button GUI element. It is registered as a basic reference and its handle can be cast to the Window and ButtonBase classes. Beside these methods it has all methods of the Window and ButtonBase classes.

\begin{titled-itemize}{Operators and methods}
  \item bool IsSelected() const -- returns true if the checkbox is selected (has the checkmark)
  \item void SetSelected(bool) -- sets whether the checkbox is selected or not
  \item uint32 GetGroupID() const -- returns the group ID assigned to this radio button
  \item void SetGroupID(uint32) -- sets the group ID for this radio button
\end{titled-itemize}

\subsubsection{Editbox}

This class representes the editbox GUI element. It is registered as a basic reference and its handle can be cast to the Window class. Beside these methods it has all methods of the Window class.

\begin{titled-itemize}{Operators and methods}
  \item bool HasInputFocus() const -- returns true if the editbox has input focus
  \item bool IsReadOnly() const -- returns true if the editbox is read-only
  \item bool IsTextMasked() const -- returns true if the text for the editbox will be rendered masked
  \item bool IsTextValid() const -- returns true if the editbox text is valid given the currently set
 validation string
  \item const CEGUIString\& GetValidationString() const -- returns the currently set validation string
  \item uint32 GetMaskCodePoint() const -- returns the utf32 code point used when rendering masked text
  \item uint32 GetMaxTextLength() const -- returns the maximum text length set for this editbox
  \item void SetReadOnly(bool) -- specifies whether the editbox is read-only
  \item void SetTextMasked(bool) -- specifies whether the text for the editbox will be rendered masked
  \item void SetValidationString(const CEGUIString \&in) -- sets the text validation string
  \item void SetMaskCodePoint(uint32) -- sets the utf32 code point used when rendering masked text
  \item void SetMaxTextLength(uint32) -- sets the maximum text length for this editbox
\end{titled-itemize}

\subsubsection{GUIMgr}

This class manages the graphic user interface of the game. It is registered as a no-handle reference so the only way to use it is to get reference from a global property.

\begin{titled-itemize}{Operators and methods}
  \item void LoadScheme(const string \&in) -- loads the project scheme file
  \item void LoadImageset(const string \&in) -- loads the project imageset file
  \item Window@ LoadLayout(const CEGUIString \&in, const CEGUIString \&in) -- loads the project layout file and returns it (the layout filename, widget names in layout file are prefixed with given prefix)
\end{titled-itemize}

\subsection{Global functions}

\begin{titled-itemize}{Declaration and comment}
  \item Window@ GetWindow(string) -- returns the window with the given name or null if such a window does not exist
  \item void Println(const T \&in) -- writes a message to the log or the console (converts all types T from PropertyTypes.h to string)
  \item const string GetTextData(const StringKey \&in, const StringKey \&in) -- returns a text data specified by a given search key and group
  \item const string GetTextData(const StringKey \&in) -- returns a text data specified by a given search key from a default group
\end{titled-itemize}

\subsection{Global properties}

\begin{titled-itemize}{Declaration and comment}
  \item EntityHandle this -- returns the handle of entity that calls the current function, invalid handle if no entity calls it
  \item EntityMgr\& gEntityMgr -- the reference to the entity manager on which it is possible to call methods, this reference cannot be stored to a variable or passed to a function
  \item InputMgr\& gInputMgr -- the reference to the input manager on which it is possible to call methods, this reference cannot be stored to a variable or passed to a function
  \item Project\& gProject -- the reference to the current project on which it is possible to call methods, this reference cannot be stored to a variable or passed to a function
  \item Game\& game -- the reference to the current game on which it is possible to call methods, this reference cannot be stored to a variable or passed to a function
  \item GUIMgr\& gGUIMgr -- the reference to the graphic user interface manager on which it is possible to call methods, this reference cannot be stored to a variable or passed to a function
\end{titled-itemize}

\subsection{Enumerations}

\begin{titled-itemize}{Type \{ values \} -- comment}
  \item eEntityMessageType \{ \ldots \} -- this is user-defined, the types are got from EntityMessageTypes.h
  \item eEntityMessageResult \{ RESULT\_IGNORED, RESULT\_OK, RE\-SULT\-\_\-ER\-ROR \} -- result receives after sending out a message to entities (message is ignored, processed well or an error occured)
  \item eEntityDescriptionKind \{ EK\_ENTITY, EK\_PROTOTYPE \} -- kind of an entity (an ordinary entity, a prototype)
  \item eComponentType \{ \ldots \} -- this is user-defined, the types are got from \_ComponentTypes.h 
  \item ePropertyAccess \{ PA\_EDIT\_READ = 1$<<$1, PA\_EDIT\_WRITE = 1$<<$2, PA\_SCRIPT\_READ = 1$<<$3, PA\_SCRIPT\_WRITE = 1$<<$4, PA\_INIT = 1$<<$5, PA\_FULL\_ACCESS = 0xff \} -- restrictions of access which can be granted to a property (the property can be read/written from editor/scripts/during the component initialization or full access is granted)
  \item eKeyCode \{ \ldots \} -- this is defined in KeyCodes.h
  \item eMouseButton \{ MBTN\_LEFT, MBTN\_RIGHT, MBTN\_MIDDLE, \\MBTN\_UNKNOWN \} -- all possible buttons of the mouse device
\end{titled-itemize}

\subsection{Typedefs}

\begin{titled-itemize}{New type = old type}
  \item float32 = float
  \item float64 = double
  \item EntityID = int32
  \item EntityTag = uint16
  \item ComponentID = int32
  \item PropertyAccessFlags = uint8
\end{titled-itemize}

\section{Additional registered objects}

This section is about additional registered objects so it should be edited when new objects are registered to the script engine.

\subsection{Classes}

\subsubsection{Point}

This class represents 2D point and it is registered as a value type.

\begin{titled-itemize}{Properties}
  \item int32 x -- x coordinate
  \item int32 y -- y coordinate
\end{titled-itemize}

\begin{titled-itemize}{Constructors}
  \item Point() -- default constructor (x = y = 0)
  \item Point(const Point \&in) -- copy constructor
  \item Point(int32 x, int32 y)  -- constructor using coordinates
\end{titled-itemize}

\begin{titled-itemize}{Operators and methods}
  \item Point operator-() const -- negate this point
  \item void Set(int32, int32) -- set this point to some specified coordinates
\end{titled-itemize}

\subsubsection{Color}

This class represents 32-bit color and it is registered as a value type.

\begin{titled-itemize}{Properties}
  \item uint8 r -- red component of color
  \item uint8 g -- green component of color
  \item uint8 b -- blue component of color
  \item uint8 a -- alpha component of color
\end{titled-itemize}

\begin{titled-itemize}{Constructors}
  \item Color() -- default constructor (r = g = b = 0, a = 255)
  \item Color(uint8 r, uint8 g, uint8 b, uint8 a = 255) -- parameter constructor
  \item Color(uint32 color) -- construct the color from 32-bit number
\end{titled-itemize}

\begin{titled-itemize}{Operators and methods}
  \item bool operator==(const Color \&in) const -- equality operator
  \item uint32 GetARGB() const -- return 32-bit representation of the color
\end{titled-itemize}

\subsection{Global functions}

\begin{titled-itemize}{Declaration and comment}
  \item float32 Random(const float32, const float32) -- returns the random real number between the first and the second parameter
  \item float32 Abs(const float32)
  \item int32 Abs(const int32)
  \item float32 Min(const float32, const float32)
  \item int32 Min(const int32, const int32)
  \item Vector2 Min(const Vector2 \&in, const Vector2 \&in)
  \item float32 Max(const float32, const float32)
  \item int32 Max(const int32, const int32)
  \item Vector2 Max(const Vector2 \&in, const Vector2 \&in)
  \item int32 Round(const float32)
  \item int64 Round(const float64)
  \item int32 Floor(const float32)
  \item int32 Ceiling(const float32)
  \item float32 Sqr(const float32)
  \item float32 Sqrt(const float32)
  \item float32 Distance(const Vector2 \&in, const Vector2 \&in)
  \item float32 DistanceSquared(const Vector2 \&in, const Vector2 \&in)
  \item float32 AngleDistance(const float32, const float32)
  \item float32 Sin(const float32)
  \item float32 Cos(const float32)
  \item float32 Tan(const float32)
  \item float32 ArcTan(const float32)
  \item float32 ArcSin(const float32)
  \item float32 Dot(const Vector2 \&in, const Vector2 \&in)
  \item float32 Cross(const Vector2 \&in, const Vector2 \&in)
  \item float32 Clamp(const float32, const float32, const float32) -- returns the first parameter if it is between the second and the third one, the second one if the first one is lesser than the second one and the third one if the first on is greater than the third one
  \item float32 ClampAngle(const float32) -- calls Clamp(param, 0, 2*PI)
  \item bool IsAngleInRange(const float32, const float32, const float32)
  \item float32 Wrap(const float32, const float32, const float32) -- substracts the difference of the third and the second parameter from the first one until it is lesser than the third one and adds the difference of the third and the second parameter to the first one until it is greater than the second one
  \item float32 WrapAngle(const float32) -- calls Wrap(param, 0, 2*PI)
  \item float32 Angle(const Vector2 \&in, const Vector2 \&in)
  \item float32 RadToDeg(const float32)
  \item Vector2 VectorFromAngle(const float32, const float32) -- creates the vector with the angle from the first parameter and the size from the second one
  \item bool IsPowerOfTwo(const uint32)
  \item float32 ComputePolygonArea(Vector2[] \&in)
  \item int32 GetState() -- returns the state of script when it is called from the OnAction() handler, an error otherwise
  \item void SetAndSleep(int32, uint64) -- sets the state of script to the first argument and the time of the next execution to the current time plus the second argument in milliseconds when it is called from the OnAction() handler, throws an error otherwise
\end{titled-itemize}

\subsection{Global properties}

\begin{titled-itemize}{Declaration and comment}
  \item float32 PI - the PI constant
\end{titled-itemize}

\end{document}

