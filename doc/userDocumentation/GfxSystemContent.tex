\begin{description}
  \item[Namespaces:] GfxSystem
  \item[Headers:]GfxRenderer.h,GfxSceneMgr.h, GfxStructures.h, GfxWindow.h, IGfxWindowListener.h, OglRenderer.h, Texture.h, RenderTarget.h, GfxViewport.h, DragDropCameraMover.h, PhysicsDraw.h
  \item[Source files:] GfxRenderer.cpp,GfxSceneMgr.cpp, GfxStructures.cpp, GfxWindow.cpp, OglRenderer.cpp, Texture.cpp, GfxViewport.cpp, DragDropCameraMover.cpp, PhysicsDraw.cpp
  \item[Classes:] GfxRenderer, OglRenderer, GfxSceneMgr, GfxWindow, IGfxWindowListener, Texture, GfxViewport, DragDropCameraMover, PhysicsDraw
  \item[Libraries used:] SDL, OpenGL, SOIL
\end{description}

\section{Introduction}

 \emph{GfxSystem} implements functionalities related to the rendering of game entities. The design of this system is influenced by the the requirement of platform independence. 

Note that \emph{GUISystem} uses its own renderering system.

\section{Application window}
\emph{GfxSystem} also manages creating and handling the application window which depends on the used operating system. In the Ocerus, this functionality is implemented by the \emph{GfxWindow} class with the usage of the SDL library.

SDL (Simple DirectMedia Layer) is a free cross-platform multi-media development API used for games, emulators, MPEG players, and other applications. The main advantage is that it supports many operating systems.

It is used for handling window events aswell. Currently, the only event that is being handled is the "change resolution" event. There is a listener implemeted for this event(\emph{IGfxWindowListener}) so every other system that needs to be informed about a resolution change can add its own listener. This feature is used by the \emph{InputSystem} and the \emph{GUISystem}.

Note that SDL also provides features in low-level audio and input management but since audio is not yet implemented and input management is done by more specialized library, the only SDL features used is Ocerus are window management and creating rendering context.

\section{Renderer}
For low-level rendering, the OpenGL library is used. OpenGL works on most platforms but not on all. \emph{GfxSystem} is designed in a way that allows to easily implement support for other low-level graphic libraries (e.g. DirectX). It is just necessary to create new derived class and implement few virtual functions. The base class that communicates with other systems is the \emph{GfxRender} and its derived class using the OpenGL calls is the \emph{OglRenderer}.

\section{Textures}
For loading the textures, SOIL library is used.
SOIL (Simple OpenGL Image Library) is a tiny C library used for uploading textures into the OpenGL. It supports most of the the common image formats.

Loading textures is implemented in the class \emph{Texture}. It inherits from  \emph{Resource} class, see \emph{ResourceSystem}. 

Since SOIL depends on the OpenGL, class \emph{Texture} uses methods from \emph{GfxRenderer} or more precisely \emph{OglRender}.
 
\section{Process of rendering the entities}
Each entity, upon its initialization, registers its graphic representing components in the \emph{GfxSceneMgr}. Upon the request from the game loop in the \emph{Core}, the \emph{GfxSceneMgr} draws all registered components, using methods from the  \emph{GfxRenderer}. It is also necessary to set up the rendering target within the \emph{GfxRenderer}. This is usually done by the \emph{GUISystem}.


%\section{Glossary}
%This is a glossary of the most used terms in the previous sections:
%
%\begin{description}
%  \item[viewport] -- the region of the application window to where the renderer draws.
%  \item[rendering target] -- the rendering target is defined by the viewport and the camera.
%\end{description}


%\begin{thebibliography}{9}                                                                                                
%\bibitem {angelscript}AngelScript documentation, file /AngelScript/index.html
%\end{thebibliography}
