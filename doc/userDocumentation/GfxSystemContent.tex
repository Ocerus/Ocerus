\begin{description}
  \item[Namespaces:] GfxSystem
  \item[Headers:] DragDropCameraMover.h, GfxRenderer.h, GfxSceneMgr.h, Gfx\-Structures.\-h, GfxViewport.h, GfxWindow.h, IGfxWindowListener.h, Mesh.h, OglRenderer.h, PhysicsDraw.h, RenderTarget.h, Texture.h
  \item[Source files:] DragDropCameraMover.cpp, GfxRenderer.cpp, Gfx\-Scene\-Mgr.\-cpp, GfxStructures.cpp, GfxViewport.cpp, GfxWindow.cpp, Mesh.cpp, OglRenderer.cpp, PhysicsDraw.cpp, Texture.cpp
  \item[Classes:] DragDropCameraMover, GfxRenderer, GfxSceneMgr, Gfx\-Viewport, GfxWindow, IGfxWindowListener, Mesh,  OglRenderer, PhysicsDraw, Texture
  \item[Libraries used:] SDL, OpenGL, SOIL, Box2D
\end{description}

\section{Purpose of the graphic system}

The graphic system implements functionalities related to the rendering of game entities and the management of the application window. The design of this system is influenced by the requirement of platform independence. Note that the GUI system uses its own rendering system.

In the following sections the concept of viewports and render targets will be described as well as the process of rendering game entities, the way of creating the application window will be revealed and the management of meshes and textures will be introduced. In the last section there is a small glossary of used terms.

\section{Graphic viewport and render target}

The \emph{GfxSystem::GfxViewport} class defines a place where all game entities will be rendered. It simply stores the information about a position and a size within the global window that can be obtained from some texture and also the data needed for drawing a grid which is useful in the edit mode. It has methods for getting and setting these properties as well as other ones for calculating its boundaries in the world or scene space.

For drawing the game entities it is also necessary to know from which position they are rendered so the \emph{GfxSyste::RenderTarget} type is defined which is a pair of a viewport and an entity handle that must point to an entity with a camera component. This type is used by renderer classes described below where it is indexed by the \emph{GfxSystem::RenderTargetID} type which is defined as an integer.

For easy moving and zooming a camera by a mouse in a render target the \emph{GfxSystem::DragDropCameraMover} class was defined. In its constructor or later by its setters it is possible to adjust a zoom sensitivity and a maximal and minimal allowed zoom.

\section{Renderer and scene manager}

The \emph{GfxSystem::GfxRenderer} is the main class that manages a rendering of entities to render targets. This is a platform independent abstract class handling a communication with other engine systems from which now derives only the \emph{GfxSystem::OglRenderer} class implementing a low level rendering in the OpenGL \cite{opengl} library. If it is necessary to implement a rendering for another library (i.e. DirectX) it should be done by deriving another class and implementing all abstract methods.

The abstract class has methods for managing its render targets, for drawing simple shapes as well as textures and meshes or for clearing the screen. The rendering must be started by the \emph{GfxRenderer::BeginRendering} method, then the current render target must be set and cleared. After everything is drawn the \emph{GfxRenderer::FinalizeRenderTarget} method must be called and then another render target is set or the whole rendering is finished by the \emph{GfxRenderer::EndRendering} method.

An important attribute of the \emph{GfxSystem::GfxRenderer} class is the point\-er to the \emph{GfxSystem::SceneMgr} class created on its initialization accessible by the \emph{GfxRenderer::GetSceneManager} method. This is the class to which all drawable components (sprites, models) must be registered along with a \emph{Transform} component of their entity by the \emph{SceneMgr::AddDrawable} method so then they are rendered by the \emph{SceneMgr::DrawVisibleDrawables} method if they are visible.

To provide debug drawing of physics entities the \emph{GfxSystem::PhysicsDraw} proxy class was defined and registered as an implementation of the \emph{b2\-Debug\-Draw} class from the Box2D library. All methods are redirected to corresponding methods in the \emph{GfxSystem::GfxRenderer} class.

\section{Application window}

The graphic system also manages creating and handling the application window which depends on the used operating system. This functionality is implemented by the \emph{GfxSystem::GfxWindow} class with the usage of the SDL library. This class has methods for getting and setting a window position, size and title or a visibility of a mouse cursor, toggling a fullscreen mode and handling system window events. It is also possible to register a screen listener represented by a class implementing the \emph{GfxSystem::IGfxWindowListener} interface. This class will be informed when the screen resolution is changed.

Note that the SDL library also provides features in low-level audio and input management but since audio is not yet implemented and input management is done by more specialized library, the only used SDL features used are window management and creating rendering context.

\section{Mesh and texture}

Meshes and textures are essential parts of the \emph{Model} and \emph{Sprite} components. They can be loaded via the \emph{GfxSystem::Mesh} and \emph{GfxSystem::Texture} classes that inherit from the \emph{ResourceSystem::Resource} class (for more information see chapter about the resource system).

On loading of a texture resource the \emph{GfxRenderer::LoadTexture} abstract method is called. For OpenGL implementation the SOIL library is used which is a tiny C library used for uploading textures into the OpenGL and which supports most of the common image formats.

For defining meshes the Wavefront OBJ file format \cite{obj} is used. Every texture used in the model definition is automatically loaded as a resource.
 
\section{Glossary}
This is a glossary of the most used terms in the previous sections:

\begin{description}
  \item[Viewport] -- a region of the application window where entities are rendered to
  \item[Render target] -- a pair or a viewport and a camera
  \item[Sprite] -- a component for showing an entity as an image (even animated or transparent)
  \item[Model] -- a component for showing an entity as a 3D-model
  \item[Texture] -- a bitmap image applied to a surface of a graphic object
  \item[Mesh] -- a collection of vertices, edges and faces that defines the shape of a polyhedral object
\end{description}
