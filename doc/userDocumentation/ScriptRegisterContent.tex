\section{Purpose of this document}

This document is an enumeration of classes, global functions etc. that are registered to the script engine so they can be used from scripts. It is divided to the section about objects from the game engine and to the section about additional user-defined objects.

\section{Integral registered objects}

This section is about registered objects from the game engine so it should not be edited until the game engine is changed.

\subsection{Classes}

\subsubsection{StringKey}

This class serves as a key into maps and other structures where we want to index data using strings, but we need high speed as well. The string value is hashed and the result is then used as a decimal representation of the string. This class is registered as a value type.

\begin{titled-itemize}{Constructors}
	\item StringKey() -- default constructor
	\item StringKey(const StringKey \&in) -- copy constructor
	\item StringKey(const string \&in) -- constructs the key from a standard string
\end{titled-itemize}

\begin{titled-itemize}{Operators and methods}
  \item bool operator=(const StringKey \&in) const -- assignment operator
  \item StringKey\& operator==(const StringKey \&in) -- equality operator
  \item string ToString() const -- converts the key to a string
\end{titled-itemize}

\subsubsection{array\_T}

This is a group of classes parametrized by a type T from PropertyTypes.h that serve as an array of values of a type T corresponding to properties of the Array$<$T$>$* type. For example an array of 32-bit signed integer will be declared as \verb/array_int32/. These classes are not compatible with an script array defined as for example \verb/int32[]/ but they have similar methods. This class is registered as a value type.

\begin{titled-itemize}{Constructors}
  \item array\_T() -- default constructor that should not be used, instances of these classes are got from Get\_array\_T and Get\_const\_array\_T methods of EntityHandle class
\end{titled-itemize}

\begin{titled-itemize}{Operators and methods}
  \item T\& operator[](int32) -- write accessor to an array item
  \item T operator[](int32) const -- read accessor to an array item
  \item int32 GetSize() const -- returns a size of the array
  \item void Resize(int32) -- resize an array to a new size
\end{titled-itemize}

\subsubsection{PropertyFunctionParameters}

This class represents generic parameters passed to a function accessed via the properties system. Thanks to the $<<$ operators the function parameter can be passed as \verb/PropertyFunctionParameters()/ \verb/<< param1 << param2 .../. This class is registered as a value type.

\begin{titled-itemize}{Constructors}
  \item PropertyFunctionParameters() -- constructs new empty parameters
\end{titled-itemize}

\begin{titled-itemize}{Operators and methods}
  \item PropertyFunctionParameters\& operator=(const PropertyFunctionParameters \&in) -- assignment operator
  \item bool operator==(const PropertyFunctionParameters \&in) const -- e\-qua\-li\-ty operator
  \item PropertyFunctionParameters operator$<<$(const T \&in) const -- add parameter of type T from PropertyTypes.h
  \item PropertyFunctionParameters operator$<<$(const array\_T \&in) const -- add parameter of type array\_T
\end{titled-itemize}

\subsubsection{EntityHandle}

This class represents one unique entity in the entity system. This class is registered as a value type.

\begin{titled-itemize}{Constructors}
  \item EntityHandle() -- default constructor will initialize the handle to an invalid state
  \item EntityHandle(const EntityHandle \&in) -- only the copy constructor is enabled, new entities should be added only by the EntityMgr
\end{titled-itemize}

\begin{titled-itemize}{Operators and methods}
  \item EntityHandle\& operator=(const EntityHandle \&in) -- assignment operator
  \item bool operator==(const EntityHandle \&in) -- equality operator
  \item bool IsValid() const -- returns true if this handle is valid (not null)
  \item bool Exists() const -- returns true if this entity still exists in the system
  \item EntityID GetID() const -- returns the internal ID of this entity
  \item string GetName() const -- returns the name of this entity
  \item EntityTag GetTag() const -- returns the tag of this entity
  \item void SetTag(EntityTag) -- sets the tag of this entity
  \item eEntityMessageResult PostMessage(const eEntityMessageType, PropertyFunctionParameters = PropertyFunctionParameters()) -- sends a message to this entity (message type, parameters)
  \item void CallFunction(string \&in, PropertyFunctionParameters \&in) -- calls a function on an entity of a specific name and with specific parameters
  \item T Get\_T(string \&in) -- gets a value of an entity property of a specific name and type T from PropertyTypes.h
  \item void Set\_T(string \&in, T) -- sets a value of an entity property of a specific name and type T from PropertyTypes.h
  \item array\_T Get\_array\_T(string \&in) -- gets a non-constant value of an entity property of a specific name and type of array\_T
  \item const array\_T Get\_const\_array\_T(string \&in) -- gets a constant value of an entity property of a specific name and type of array\_T
  \item bool RegisterDynamicProperty\_T(const string \&in, const PropertyAccessFlags, const string \&in) -- registers an entity dynamic property of a specific name, access and comment
  \item bool UnregisterDynamicProperty(const string \&in) const -- unregisters an entity dynamic property of a specific name
\end{titled-itemize}

\subsubsection{Vector2}

This class represents 2D vector and it is registered as a value type.

\begin{titled-itemize}{Properties}
  \item float32 x -- x coordinate
  \item float32 y -- y coordinate
\end{titled-itemize}

\begin{titled-itemize}{Constructors}
  \item Vector2() -- default constructor (x = y = 0)
  \item Vector2(const Vector2 \&in) -- copy constructor
  \item Vector2(float32 x, float32 y)  -- constructor using coordinates
\end{titled-itemize}

\begin{titled-itemize}{Operators and methods}
  \item void operator+=(const Vector2 \&in) -- add a vector to this vector
  \item void operator-=(const Vector2 \&in) -- subtract a vector from this vector
  \item void operator*=(float32) -- multiply this vector by a scalar
  \item Vector2 operator-() const -- negate this vector
  \item bool operator==(const Vector2 \&in) const -- equality operator
  \item Vector2 operator+(const Vector2 \&in) const -- returns a sum of this and argument vector
  \item Vector2 operator-(const Vector2 \&in) const -- returns a difference of this and argument vector
  \item Vector2 operator*(float32) const -- returns a product of a scalar and this vector
  \item float32 Length() const -- get the length of this vector (the norm)
  \item float32 LengthSquared() const -- get the length squared
  \item void Set(float32, float32) -- set this vector to some specified coordinates
  \item void SetZero() -- set this vector to all zeros
  \item float32 Normalize() -- converts this vector into a unit vector and returns the length
  \item bool IsValid() const -- returns whether this vector contain finite coordinates
  \item float32 Dot(const Vector2 \&in) const -- returns a scalar product of this and argument vector
\end{titled-itemize}

\subsubsection{EntityPicker}

This class allows the game to pick an entity based on provided input data. This class is registered as a value type.

\begin{titled-itemize}{Constructors}
  \item EntityPicker(const Vector2 \&in, const int32, const int32) -- creates a picker for selecting an entity under the current mouse cursor
\end{titled-itemize}

\begin{titled-itemize}{Operators and methods}
  \item EntityHandle PickSingleEntity() -- runs the picking query, the result is returned directly
  \item void PickMultipleEntities(EntityHandle[] \&out, const Vector2 \&in, const float32) -- runs the picking query, the result is filled into the given array, the query is defined by a rectangle between the last cursor position and the given cursor position, the rectangle is rotated by the given angle
\end{titled-itemize}

\subsubsection{EntityDescription}

This class contains all info needed to create one instance of an entity. It is basically a collection of component descriptions. This class is registered as a value type.

\begin{titled-itemize}{Constructors}
  \item EntityDescription() -- default constructor
\end{titled-itemize}

\begin{titled-itemize}{Operators and methods}
  \item void Reset() -- clears everything, call this before each subsequent filling of the description
  \item void AddComponent(const eComponentType) -- adds a new component specified by its type
  \item void SetName(const string \&in) -- sets a custom name for this entity
  \item void SetKind(const eEntityDescriptionKind) -- sets this entity to be an ordinary entity or a prototype
  \item void SetPrototype(const EntityHandle) -- sets the prototype the entity is to be linked to
  \item void SetPrototype(const EntityID) -- sets the prototype the entity is to be linked to
  \item void SetDesiredID(const EntityID) -- sets a desired ID for this entity, it does not have to be used by EntityMgr
\end{titled-itemize}

\subsubsection{EntityMgr}

This class manages all game entities. It is registered as a no-handle reference so the only way to use it is to get reference from a global property.

\begin{titled-itemize}{Operators and methods}
  \item EntityHandle CreateEntity(EntityDescription \&in) -- creates a new entity accordingly to its description and returns its handle
  \item EntityHandle InstantiatePrototype(const EntityHandle, const string \&in) -- creates a new entity from a prototype with a specific name
  \item EntityHandle DuplicateEntity(const EntityHandle, const string \&in) -- duplicates an entity with a specific new name
  \item void DestroyEntity(const EntityHandle) -- destroys a specified entity if it exists
  \item bool EntityExists(const EntityHandle) const -- returns true if the entity exists
  \item EntityHandle FindFirstEntity(const string \&in) -- returns EntityHandle to the first entity of a specified name
  \item EntityHandle GetEntity(EntityID) const -- return EntityHandle of the entity with specified ID
  \item bool IsEntityInited(const EntityHandle) const -- returns true if the entity was fully initialized
  \item bool IsEntityPrototype(const EntityHandle) const -- returns true if the entity is a prototype
  \item void LinkEntityToPrototype(const EntityHandle, const EntityHandle) -- assigns the given entity to the prototype
  \item void UnlinkEntityFromPrototype(const EntityHandle) -- destroys the link between the component and its prototype
  \item bool IsPrototypePropertyShared(const EntityHandle, const String\-Key) const -- returns true if the property of the prototype is marked as shared (and thus propagated to instances)
  \item void SetPrototypePropertyShared(const EntityHandle, const String\-Key) -- marks the property as shared among instances of the prototype
  \item void SetPrototypePropertyNonShared(const EntityHandle, const String\-Key) -- marks the property as non shared among instances of the prototype
  \item void UpdatePrototypeInstances(const EntityHandle) -- propagates the current state of properties of the prototype to its instances
  \item bool HasEntityProperty(const EntityHandle, const StringKey, const PropertyAccessFlags) const -- returns true if the entity has the given property
  \item bool HasEntityComponentProperty(const EntityHandle, const ComponentID, const StringKey, const PropertyAccessFlags) const -- returns true if the component of the entity has the given property
  \item void BroadcastMessage(const eEntityMessageType, PropertyFunctionParameters = PropertyFunctionParameters()) -- sends a message to all entities
  \item bool HasEntityComponentOfType(const EntityHandle, const eComponentType) -- returns true if the given entity has a component of the given type
  \item int32 GetNumberOfEntityComponents(const EntityHandle) const -- returns the number of components attached to the entity
  \item ComponentID AddComponentToEntity(const EntityHandle, const e\-Com\-po\-nent\-Ty\-pe) -- adds a component of the specified type to the entity, returns an ID of the new component
  \item void DestroyEntityComponent(const EntityHandle, const Com\-po\-nent\-ID) -- destroys a component of the entity
\end{titled-itemize}

\subsubsection{MouseState}

This structure contains the state of the mouse device at a specific point of time. This class is registered as a simple value type.

\begin{titled-itemize}{Properties}
  \item int32 x -- x coordinate
  \item int32 y -- y coordinate
  \item int32 wheel -- wheel position
  \item uint8 buttons -- pressed buttons
\end{titled-itemize}

\subsubsection{InputMgr}

This class manages the game input. It is registered as a no-handle reference so the only way to use it is to get reference from a global property.

\begin{titled-itemize}{Operators and methods}
  \item void CaptureInput() -- updates the state of the manager and processes all events
  \item bool IsKeyDown(const eKeyCode) const -- returns true if a specified key is down
  \item bool IsMouseButtonPressed(const eMouseButton) const -- returns true if a specified button of the mouse is pressed
  \item MouseState\& GetMouseState() const -- returns the current state of the mouse
\end{titled-itemize}

\subsubsection{Project}

This class represents the current project. It is registered as a no-handle reference so the only way to use it is to get reference from a global property.

\begin{titled-itemize}{Operators and methods}
  \item bool OpenScene(const string \&in) -- opens the scene with given filename
  \item bool OpenSceneAtIndex(int32) -- opens the scene at the given index in the scene list
  \item uint32 GetSceneCount() const -- returns the number of scenes
  \item int32 GetSceneIndex(const string \&in) const -- returns the index of the scene with a specific name, -1 if does not exist
  \item string GetOpenedSceneName() const -- returns name of the opened scene, or empty string if no scene is opened
  \item string GetSceneName(int32) const -- returns name of the scene at the given index in the scene list
\end{titled-itemize}

\subsubsection{Game}

This class represents the current game. It is registered as a no-handle reference so the only way to use it is to get reference from a global property.

\begin{titled-itemize}{Operators and methods}
  \item void ClearDynamicProperties() -- clears the dynamic property list
  \item bool HasDynamicProperty(const string \&in) const -- returns whether the specified dynamic property exists
  \item bool DeleteDynamicProperty(const string \&in) -- deletes the specified dynamic property
  \item bool LoadFromFile(const string \&in) -- loads the game from the specified file
  \item void PauseAction() -- pauses the game action until resumed again
  \item void ResumeAction() -- resumes the game action if paused
  \item bool SaveToFile(const string \&in) -- saves the game to the specified file
  \item void Quit() -- quits the game
  \item uint64 GetTime() -- returns the current game time in miliseconds
  \item bool GetFullscreen() -- returns whether the game is in the fullscreen mode
  \item void SetFullscreen(const bool) -- sets whether the game is in the fullscreen mode
  \item T Get\_T(string \&in) -- gets a value of a dynamic property of a specific name and type T from PropertyTypes.h
  \item void Set\_T(string \&in, T) -- sets a value of a dynamic property of a specific name and type T from PropertyTypes.h
\end{titled-itemize}

\subsubsection{CEGUIString}

This class represents string in the CEGUI library. It is registered as a value type and it is implicitly cast from/to a common string so no other methods are needed.

\subsubsection{Window}

This class is the base class for all GUI elements. It is registered as a basic reference and its handle can be cast to all of its desceants.

\begin{titled-itemize}{Operators and methods}
  \item const CEGUIString\& GetName() const -- returns the name of this window
  \item const CEGUIString\& GetType() const -- returns the type name for this window
  \item bool IsDisabled() const -- returns whether the window is currently disabled (does not inherit state from ancestor windows)
  \item bool IsDisabled(bool) const -- returns whether the window is currently disabled (specify whether to inherit state from ancestor windows)
  \item bool IsVisible() const -- returns true if the window is currently visible (does not inherit state from ancestor windows)
  \item bool IsVisible(bool) const -- returns true if the window is currently visible (specify whether to inherit state from ancestor windows)
  \item bool IsActive() const -- returns true if this is the active window (may receive user inputs)
  \item const CEGUIString\& GetText() const -- returns the current text for the window
  \item bool InheritsAlpha() const -- returns true if the window inherits alpha from its parent(s)
  \item float32 GetAlpha() const -- returns the current alpha value set for this window (between 0.0 and 1.0)
  \item float32 GetEffectiveAlpha() const -- returns the effective alpha value that will be used when rendering this
 window
  \item Window\@ GetParent() const -- returns the parent of this window
  \item const CEGUIString\& GetTooltipText() const -- returns the current tooltip text set for this window
  \item bool InheritsTooltipText() const -- returns whether this window inherits tooltip text from its parent when
 its own tooltip text is not set
  \item void SetEnabled(bool) -- sets whether this window is enabled or disabled
  \item void SetVisible(bool) -- sets whether this window is visible or hidden
  \item void Activate() -- activates the window giving it input focus and bringing it to the top of
 all windows
  \item void Deactivate() -- deactivates the window
  \item void SetText(const CEGUIString\& in) -- sets the current text string for the window
  \item void SetAlpha(float32) -- sets the current alpha value for this window
  \item void SetInheritsAlpha(bool) -- sets whether this window will inherit alpha from its parent windows
  \item void SetTooltipText(const CEGUIString\& in) -- sets the tooltip text for this window
  \item void SetInheritsTooltipText(bool) -- sets whether this window inherits tooltip text from its parent when its
 own tooltip text is not set
\end{titled-itemize}

\subsubsection{ButtonBase}

This class is the base class for all button GUI elements. It is registered as a basic reference and its handle can be cast to all of its desceants and the Window class. Beside these methods it has all methods of the Window class.

\begin{titled-itemize}{Operators and methods}
  \item bool IsHovering() const -- returns true if user is hovering over this widget
  \item bool IsPushed() const -- returns true if the button widget is in the pushed state
\end{titled-itemize}

\subsubsection{Checkbox}

This class representes the checkbox GUI element. It is registered as a basic reference and its handle can be cast to the Window and ButtonBase classes. Beside these methods it has all methods of the Window and ButtonBase classes.

\begin{titled-itemize}{Operators and methods}
  \item bool IsSelected() const -- returns true if the checkbox is selected (has the checkmark)
  \item void SetSelected(bool) -- sets whether the checkbox is selected or not
\end{titled-itemize}

\subsubsection{PushButton}

This class representes the push button GUI element. It is registered as a basic reference and its handle can be cast to the Window and ButtonBase classes. It has all methods of the Window and ButtonBase classes and none more.

\subsubsection{RadioButton}

This class representes the radio button GUI element. It is registered as a basic reference and its handle can be cast to the Window and ButtonBase classes. Beside these methods it has all methods of the Window and ButtonBase classes.

\begin{titled-itemize}{Operators and methods}
  \item bool IsSelected() const -- returns true if the checkbox is selected (has the checkmark)
  \item void SetSelected(bool) -- sets whether the checkbox is selected or not
  \item uint32 GetGroupID() const -- returns the group ID assigned to this radio button
  \item void SetGroupID(uint32) -- sets the group ID for this radio button
\end{titled-itemize}

\subsubsection{Editbox}

This class representes the editbox GUI element. It is registered as a basic reference and its handle can be cast to the Window class. Beside these methods it has all methods of the Window class.

\begin{titled-itemize}{Operators and methods}
  \item bool HasInputFocus() const -- returns true if the editbox has input focus
  \item bool IsReadOnly() const -- returns true if the editbox is read-only
  \item bool IsTextMasked() const -- returns true if the text for the editbox will be rendered masked
  \item bool IsTextValid() const -- returns true if the editbox text is valid given the currently set
 validation string
  \item const CEGUIString\& GetValidationString() const -- returns the currently set validation string
  \item uint32 GetMaskCodePoint() const -- returns the utf32 code point used when rendering masked text
  \item uint32 GetMaxTextLength() const -- returns the maximum text length set for this editbox
  \item void SetReadOnly(bool) -- specifies whether the editbox is read-only
  \item void SetTextMasked(bool) -- specifies whether the text for the editbox will be rendered masked
  \item void SetValidationString(const CEGUIString \&in) -- sets the text validation string
  \item void SetMaskCodePoint(uint32) -- sets the utf32 code point used when rendering masked text
  \item void SetMaxTextLength(uint32) -- sets the maximum text length for this editbox
\end{titled-itemize}

\subsubsection{GUIMgr}

This class manages the graphic user interface of the game. It is registered as a no-handle reference so the only way to use it is to get reference from a global property.

\begin{titled-itemize}{Operators and methods}
  \item void LoadScheme(const string \&in) -- loads the project scheme file
  \item void LoadImageset(const string \&in) -- loads the project imageset file
  \item Window@ LoadLayout(const CEGUIString \&in, const CEGUIString \&in) -- loads the project layout file and returns it (the layout filename, widget names in layout file are prefixed with given prefix)
\end{titled-itemize}

\subsection{Global functions}

\begin{titled-itemize}{Declaration and comment}
  \item Window@ GetWindow(string) -- returns the window with the given name or null if such a window does not exist
  \item void Println(const T \&in) -- writes a message to the log or the console (converts all types T from PropertyTypes.h to string)
  \item const string GetTextData(const StringKey \&in, const StringKey \&in) -- returns a text data specified by a given search key and group
  \item const string GetTextData(const StringKey \&in) -- returns a text data specified by a given search key from a default group
\end{titled-itemize}

\subsection{Global properties}

\begin{titled-itemize}{Declaration and comment}
  \item EntityHandle this -- returns the handle of entity that calls the current function, invalid handle if no entity calls it
  \item EntityMgr\& gEntityMgr -- the reference to the entity manager on which it is possible to call methods, this reference cannot be stored to a variable or passed to a function
  \item InputMgr\& gInputMgr -- the reference to the input manager on which it is possible to call methods, this reference cannot be stored to a variable or passed to a function
  \item Project\& gProject -- the reference to the current project on which it is possible to call methods, this reference cannot be stored to a variable or passed to a function
  \item Game\& game -- the reference to the current game on which it is possible to call methods, this reference cannot be stored to a variable or passed to a function
  \item GUIMgr\& gGUIMgr -- the reference to the graphic user interface manager on which it is possible to call methods, this reference cannot be stored to a variable or passed to a function
\end{titled-itemize}

\subsection{Enumerations}

\begin{titled-itemize}{Type \{ values \} -- comment}
  \item eEntityMessageType \{ \ldots \} -- this is user-defined, the types are got from EntityMessageTypes.h
  \item eEntityMessageResult \{ RESULT\_IGNORED, RESULT\_OK, RE\-SULT\-\_\-ER\-ROR \} -- result receives after sending out a message to entities (message is ignored, processed well or an error occured)
  \item eEntityDescriptionKind \{ EK\_ENTITY, EK\_PROTOTYPE \} -- kind of an entity (an ordinary entity, a prototype)
  \item eComponentType \{ \ldots \} -- this is user-defined, the types are got from \_ComponentTypes.h 
  \item ePropertyAccess \{ PA\_EDIT\_READ = 1$<<$1, PA\_EDIT\_WRITE = 1$<<$2, PA\_SCRIPT\_READ = 1$<<$3, PA\_SCRIPT\_WRITE = 1$<<$4, PA\_INIT = 1$<<$5, PA\_FULL\_ACCESS = 0xff \} -- restrictions of access which can be granted to a property (the property can be read/written from editor/scripts/during the component initialization or full access is granted)
  \item eKeyCode \{ \ldots \} -- this is defined in KeyCodes.h
  \item eMouseButton \{ MBTN\_LEFT, MBTN\_RIGHT, MBTN\_MIDDLE, \\MBTN\_UNKNOWN \} -- all possible buttons of the mouse device
\end{titled-itemize}

\subsection{Typedefs}

\begin{titled-itemize}{New type = old type}
  \item float32 = float
  \item float64 = double
  \item EntityID = int32
  \item EntityTag = uint16
  \item ComponentID = int32
  \item PropertyAccessFlags = uint8
\end{titled-itemize}

\section{Additional registered objects}

This section is about additional registered objects so it should be edited when new objects are registered to the script engine.

\subsection{Classes}

\subsubsection{Point}

This class represents 2D point and it is registered as a value type.

\begin{titled-itemize}{Properties}
  \item int32 x -- x coordinate
  \item int32 y -- y coordinate
\end{titled-itemize}

\begin{titled-itemize}{Constructors}
  \item Point() -- default constructor (x = y = 0)
  \item Point(const Point \&in) -- copy constructor
  \item Point(int32 x, int32 y)  -- constructor using coordinates
\end{titled-itemize}

\begin{titled-itemize}{Operators and methods}
  \item Point operator-() const -- negate this point
  \item void Set(int32, int32) -- set this point to some specified coordinates
\end{titled-itemize}

\subsubsection{Color}

This class represents 32-bit color and it is registered as a value type.

\begin{titled-itemize}{Properties}
  \item uint8 r -- red component of color
  \item uint8 g -- green component of color
  \item uint8 b -- blue component of color
  \item uint8 a -- alpha component of color
\end{titled-itemize}

\begin{titled-itemize}{Constructors}
  \item Color() -- default constructor (r = g = b = 0, a = 255)
  \item Color(uint8 r, uint8 g, uint8 b, uint8 a = 255) -- parameter constructor
  \item Color(uint32 color) -- construct the color from 32-bit number
\end{titled-itemize}

\begin{titled-itemize}{Operators and methods}
  \item bool operator==(const Color \&in) const -- equality operator
  \item uint32 GetARGB() const -- return 32-bit representation of the color
\end{titled-itemize}

\subsection{Global functions}

\begin{titled-itemize}{Declaration and comment}
  \item float32 Random(const float32, const float32) -- returns the random real number between the first and the second parameter
  \item float32 Abs(const float32)
  \item int32 Abs(const int32)
  \item float32 Min(const float32, const float32)
  \item int32 Min(const int32, const int32)
  \item Vector2 Min(const Vector2 \&in, const Vector2 \&in)
  \item float32 Max(const float32, const float32)
  \item int32 Max(const int32, const int32)
  \item Vector2 Max(const Vector2 \&in, const Vector2 \&in)
  \item int32 Round(const float32)
  \item int64 Round(const float64)
  \item int32 Floor(const float32)
  \item int32 Ceiling(const float32)
  \item float32 Sqr(const float32)
  \item float32 Sqrt(const float32)
  \item float32 Distance(const Vector2 \&in, const Vector2 \&in)
  \item float32 DistanceSquared(const Vector2 \&in, const Vector2 \&in)
  \item float32 AngleDistance(const float32, const float32)
  \item float32 Sin(const float32)
  \item float32 Cos(const float32)
  \item float32 Tan(const float32)
  \item float32 ArcTan(const float32)
  \item float32 ArcSin(const float32)
  \item float32 Dot(const Vector2 \&in, const Vector2 \&in)
  \item float32 Cross(const Vector2 \&in, const Vector2 \&in)
  \item float32 Clamp(const float32, const float32, const float32) -- returns the first parameter if it is between the second and the third one, the second one if the first one is lesser than the second one and the third one if the first on is greater than the third one
  \item float32 ClampAngle(const float32) -- calls Clamp(param, 0, 2*PI)
  \item bool IsAngleInRange(const float32, const float32, const float32)
  \item float32 Wrap(const float32, const float32, const float32) -- substracts the difference of the third and the second parameter from the first one until it is lesser than the third one and adds the difference of the third and the second parameter to the first one until it is greater than the second one
  \item float32 WrapAngle(const float32) -- calls Wrap(param, 0, 2*PI)
  \item float32 Angle(const Vector2 \&in, const Vector2 \&in)
  \item float32 RadToDeg(const float32)
  \item Vector2 VectorFromAngle(const float32, const float32) -- creates the vector with the angle from the first parameter and the size from the second one
  \item bool IsPowerOfTwo(const uint32)
  \item float32 ComputePolygonArea(Vector2[] \&in)
  \item int32 GetState() -- returns the state of script when it is called from the OnAction() handler, an error otherwise
  \item void SetAndSleep(int32, uint64) -- sets the state of script to the first argument and the time of the next execution to the current time plus the second argument in milliseconds when it is called from the OnAction() handler, throws an error otherwise
\end{titled-itemize}

\subsection{Global properties}

\begin{titled-itemize}{Declaration and comment}
  \item float32 PI - the PI constant
\end{titled-itemize}