\begin{description}
  \item[Namespaces:] ResourceSystem
  \item[Headers:] IResourceLoadingListener.h, Resource.h, ResourceMgr.h, Re\-sour\-ce\-Ty\-pes.h, UnknownResource.h, XMLOutput.h, XMLResource.h
  \item[Source files:] Resource.cpp, ResourceMgr.cpp, ResourceTypes.cpp, XML\-Out\-put.cpp, XMLResource.cpp
  \item[Classes:] Resource, ResourceMgr, ResourcePtr, UnknownResource, XML\-Out\-put, XMLResource
  \item[Libraries used:] Boost, Expat
\end{description}

\section{Purpose of the resource manager}

Every game needs to load packs of data from external devices such as the hard drive or network. The data come in blocks belonging together and representing a unit of something usually called \emph{resource}. Since the games work with loads of resources it is necessary to organize them both in-game and on the disk. Also, the data loaded are usually quite large and it's necessary to free them when possible to save memory. All of these tasks are implemented in the resource system.

In the following sections the mapping of a resource to the engine will be described as well as the resource manager. Also a way of loading and saving scenes will be introduced. In the last section there is a small glossary of used terms.

\section{Resources}

The resource is a group of data belonging together. In the engine this is represented by the abstract \emph{ResourceSystem::Resource} class. It contains all the basic attributes of a resource and allows its users to load or unload it, but the actual implementation depends on the specific type of the resource. For example, an XML file is loaded and parsed in a different way then an OpenGL texture. However, for the user it is never really necessary to know what type of the resource he is working with and so using this class as an abstraction is enough. Only the endpoint subsystem needs to work with the specific type to be able to grab the parsed data out of it. For example, the texture resource can be carried around the system as a common \emph{ResourceSystem::Resource} class until it reaches the graphical subsystem which converts it to the texture resource and grabs the implementation specific texture data out of it.

\subsection{Resource states}

Each resource can be in one of the states described in the table \ref{tab:resource-states} at the given point of time:
\begin{table}[htbp]
	\centering
	{\scriptsize 
	\begin{tabular}{|p{0.13\hsize}|p{0.58\hsize}|}
	  \hline
	  A state & A description \\
		\hline
		Uninitialized & the system does not know about it; it's not registered \\
		Initialized & the system knows about it, but the data are not loaded yet \\
		Unloading & the data are just being unloaded \\
		Loading & the data are just being loaded \\
		Loaded & the resource is fully ready to be used \\
		Failed & the resource failed to load \\
		\hline
	\end{tabular}
	}
	\caption[Possible resource states]{Possible resource states and their description}
	\label{tab:resource-states}
\end{table}

\subsection{Content of the resource}

Once the data for the resource are loaded it must be parsed into the desired format. This can mean a data structure stored directly inside the resource class or just a handle to the data stored in other parts of the system. However, both of these must exist only in single instance in the whole system -- in the resource which parsed the data. Otherwise the data could become desynchronized. If the resource was unloaded, a pointer to its data could still exist somewhere. 

For example, the XML resource creates a tree structure for the parsed data and allows its users to traverse the tree. But nowhere in the system exists a pointer to the same tree or any of its parts. Another example is a texture. After it loads the data they are passed into the graphical subsystem which creates a platform specific texture out of it and returns only a the texture handle. The handle is then stored only in the resource.

\subsection{Resource pointers}

Since the resources are passed all around the game system we must somehow prevent any memory leaks from appearing. Doing so is quite easy however -- we simply use the shared pointer mechanism. It points to the common abstract resource (the \emph{ResourceSystem::ResourcePtr} class) and to specific resources as well (the \emph{ScriptSystem::ScriptResourcePtr} or the \emph{ResourceSystem::XMLResourcePtr} classes for example). The abstract resource pointer can be automatically converted to any specific resource pointer, but if the type does not match an assertion fault will be raised to prevent a memory corruption. All resource pointers are defined in the \verb'Utils/ResourcePointers.h' file.

\subsection{Adding custom resource types}
\label{sub:resource-adding}

There are two steps for adding a new resource type. The first one is creating a class derived from the \emph{ResourceSystem::Resource} class implementing all abstract methods (\emph{Resource::LoadImpl} and \emph{Resource::UnloadImpl}) and providing accessors to the parsed data. The methods \emph{Resource::Open/Close\-Input\-Stream} or \emph{Resource::GetRawInputData} can help with this task. Each accessing method must call the \emph{Resource::EnsureLoaded} method to make sure the resource is actually loaded before it is to be used. The second step is adding the corresponding shared pointer to the new class into the \verb'Utils/ResourcePointers.h' file.

To see all currently existing types of resources it is best to locate the \emph{ResourceSystem::Resource} class and see all classes derived from it.

\section{Resource manager}

The resource manager represented by the \emph{ResourceSystem::ResourceMgr} class takes care of organizing resources into groups and providing and interface to other parts of the system to control or grab the resources. Coupling resources into groups makes it easier to load or unload a whole bunch of them. There are methods for adding one file or a whole directory to a resource group or for getting a resource pointer to a resource of a specific name from a specific group.

To make game development easier, the source of each resource is automatically checked for an update. If it has changed, the resource is automatically reloaded if it was previously loaded. So, for example, if the user changes a currently loaded texture in an image editor and saves it it will be immediately updated in the game.

Since gaming systems have limited memory it is possible to limit the memory used by resources. Resources usually take the biggest chunk of memory, so when lowering the memory usage it is best to start here. Hopefully, the resource manager allows to define a limit which it will try to keep. When the memory is running out, it will attempt to unload resources which were not used for a long time. These resources will remain in the system and will be ready to be loaded as soon as they are needed. However, there are certain circumstances under which the resources must not be unloaded. An example of such is rendering -- no texture can be unloaded until the frame is ready. For this reason the unloading can be temporarily disabled.

If it is necessary to track the resource loading progress (for example when a scene is loading), there is the \emph{ResourceSystem::IResourceLoadingListener} interface that should be implemented and registered by the \emph{ResourceMgr\-::\-Set\-Loading\-Listener} method. Its methods are called when the loading of one resource or a resource group starts and ends.

\section{Loading and saving scenes}

For storing a state of a scene from the editor or from the game itself to a file an XML format is used. There are two classes that help with saving data to a file and loading them back.

The \emph{ResourceSystem::XMLOutput} class provides a formatted XML output to a file. First when the instance is created a file specified in a constructor parameter is opened. Then some methods for writing XML elements and attributes are used. Finally the file is closed in the destructor or in the method \emph{XMLOutput::CloseAndReport}. For example the file named \verb/file.xml/ with the XML structure

\begin{verbatim}
<?xml version="1.0" encoding="UTF-8"?>
<name key="value">
  <element>text</element>
</name>
\end{verbatim}

\noindent is created by calling this sequence of orders:

\begin{verbatim}
ResourceSystem::XMLOutput out("file.xml"); // writes header
out.BeginElementStart("name"); // writes <name
out.AddAttribute("key", "value"); // writes key="value"
out.BeginElementFinish(); // writes >
out.BeginElement("element"); // writes <element>
out.WriteString("text"); // writes text
out.EndElement(); // writes </element>
// destructor automatically closes all open elements
\end{verbatim}

\noindent As the example shows the class does an indent, remembers names of open elements and automatically closes all open elements in the end.

The \emph{ResourceSystem::XMLResource} class on the other hand loads an XML file and iterates over its elements and attributes. Since this class derives from the \emph{ResourceSystem::Resource} class first a resource pointer to the file must be get by \emph{ResourceMgr::GetResource} method and retyped to the \emph{ResourceSystem::XMLResourcePtr}. Then the top level elements can be iterated by \emph{XMLResource::IterateTopLevel} method which returns \emph{ResourceSystem::XMLNodeIterator} class which serves as an iterator and which should be compared with a result of the \emph{XMLResource::EndTopLevel} method for a detection of the end of top level elements. The iterator class has analogical methods (\emph{XMLNodeIterator::IterateChildren} and \emph{XMLNodeIterator::EndChildren}) for an iteration over children of the element it represents. Beside them it has also methods for getting an element's value, a value of its child or of its specific attribute. Since these methods are templates every value can be converted from a string to any chosen data type.

\section{Glossary}
This is a glossary of the most used terms in the previous sections:

\begin{description}
  \item[Resource] -- a unit of data usually stored in an external device the game will be working with as a whole
  \item[Resource pointer] -- a shared pointer to a resource that can be used in the whole engine
  \item[Resource type] -- resources with a common type are loaded in the same way, i.e. textures, models, scripts, texts\ldots
  \item[XML] -- Extensible Markup Language (XML) is a set of rules for encoding documents in machine-readable form
\end{description}