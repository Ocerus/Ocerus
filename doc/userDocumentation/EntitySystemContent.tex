\begin{description}
  \item[Namespaces:] EntitySystem, EntityComponents
  \item[Headers:] Component.h, ComponentEnums.h, ComponentHeaders.h, ComponentID.h, ComponentIterators.h, ComponentMgr.h, Com\-po\-nent\-Ty\-pes\-.h, EntityDescription.h, EntityHandle.h, EntityMessage.h, Entity\-Message\-Types.h, EntityMgr.h, EntityPicker.h
  \item[Source files:] Component.cpp, ComponentEnums.cpp, ComponentMgr.cpp, EntityDescription.cpp, EntityHandle.cpp, EntityMessage.cpp, Entity\-Mgr.cpp, EntityPicker.cpp
  \item[Classes:] Component, EntityComponentsIterator, ComponentMgr, EntityDescription, EntityHandle, EntityMessage, EntityMgr, EntityPicker
  \item[Libraries used:] none
\end{description}

\section{Purpose of the entity system}

The entity system creates a common interface for a definition of all game objects such as a game environment, a player character, a camera etc. and their behavior such as a drawing on a screen, an interaction with other objects etc. The object creation is based on a composition of simple functionalities that can be reused in many of them. The advantage of this unified system is an easy creating and editing of new objects from the game editor or from scripts, the disadvantage is a slower access to the object properties and behavior. It cooperates with the other systems like the graphics one for displaying objects or the script one for an interaction from scripts.

In the following sections the system of components and entities will be described as well as the extending of the system which will be probably the first action when creating a new game. In the last section there is a small glossary of used terms.

\section{Components and entities}

Every game object is represented by an entity which is a compound of components that provide it various functionalities. A component can have several properties (and functions) which can be read or written (called) via their getters and setters (or functions themselves) and which are accessible through their unique name. It can also react to sent messages such as an initialization, a drawing, a logic update etc. by its own behavior. Component properties and behaviors are accessible only through an owner entity, so it is possible to read or write a specific property of an entity if it contains a component with this property and it is also possible to send a message to an entity which dispatches it to all its components that can react on it.

\subsection{Components and their manager}

The \emph{EntitySystem::Component} class is a base class for all components used in the entity system. It inherits from the \emph{Reflection::RTTIBaseClass} class which provides the methods for working with RTTI (registering properties and functions of component). It has methods for getting the owner entity, the component type (defined in ComponentEnums.h) and the component property from its name and for posting message to the owner entity. It also introduces methods that should be overridden by specific components used for handling messages and the component creation and destruction (see section \ref{sub:entity-newcomponents}).

The \emph{EntitySystem::ComponentMgr} is a singleton class that manages instances of all entity components in the entity system. Internally it stores mapping from all entities to lists of their components. It provides methods for adding a new component of a certain type to an entity and listing or deleting all or specific components from an entity. For passing all components of an entity the \emph{EntitySystem::EntityComponentsIterator} iterator is used that encapsulates a standard iterator (for example it has the \emph{HasMore} method which returns whether the iterator is at the end of the component list).

\subsection{Entities and their manager}

An entity is represented by the \emph{EntitySystem::EntityHandle} class which sto\-res only an ID of the entity and provides methods that mostly calls corresponding methods of the entity manager with its ID. This class has also static methods that ensures all IDs in the system are unique.

For the creation of one entity the \emph{EntitySystem::EntityDescription} class is used that is basically a collection of component types. There are methods for adding a component type and setting a name and a prototype of the entity. It is also possible to set if the created object will be an instance or a prototype of an entity. Prototypes of entities are used to propagate changes of their shared properties to the instances that are linked to them so it is possible to change properties of many entities at once. Instances must have all components that has their prototype in the same order but they can also have own additional components that must be added after the compulsory ones.

It is possible to send messages to entities so there is the \emph{EntitySystem::EntityMessage} structure that represents them. It consists of the message type defined in EntityMessageTypes.h and the message parameters that are an instance of the \emph{Reflection::PropertyFunctionParameters} class. To add an parameter of any type defined in PropertyTypes.h the \emph{PushParameter} method can be called with a value as first argument or the \emph{operator}$<<$ can be used. There is also a method that checks whether the actual parameters are of the correct types according to the definition of message type (see section \ref{sub:entity-newmessages} for more information).

All entities are managed by the \emph{EntitySystem::EntityMgr} class that stores necessary information about them in maps indexed by their ID. The most of its methods has the entity handle as the first parameter that means it applies on the entity of the ID got from the handle. There are methods for creating entities from an entity description or an XML resource and for destroying them. Another methods manages entity prototypes - it is possible to link/unlink an instance to/from a prototype, to set a property as (non)shared and to invoke an update of instances of a specific prototype. Finally there are methods for getting entity properties even of a specific component (in case of two or more properties of a same name in different components), for posting and broadcasting messages to entities and for adding, listing and removing components of a specific entity.

% dokumentace tridy EntityPicker

\section{Extending the entity system}

The entity system will be probably the first system to extend when creating a new game. It is necessary to create new components with their specific properties and behavior from which new entities can be created in the game editor or from scripts. It is also possible to create new message types that can be sent to entities when it is needed to inform about new different events.

\subsection{Creating new components}
\label{sub:entity-newcomponents}

There are several steps that leads to creating a new component. First it is needed to create a class \emph{ComponentName} (to be replaced by a real component name) which publicly inherits from the \emph{Reflection::RTTIGlue$<$Component\-Name, Component$>$} class and which is in the \emph{EntityComponents} namespace. This class should override the \emph{Create}, \emph{Destroy} and \emph{HandleMessage} methods for a custom behavior on creation, destruction and handling messages. In the last method the message structure is got from the first parameter so it is possible to get a message type and message parameters (the \emph{GetParameter(index)} method is used). The method should return \emph{EntityMessage::RESULT\_OK} if the message was processed, \emph{EntityMessage::RESULT\_ IGNORED} if it was ignored or \emph{EntityMessage::RESULT\_ERROR} if an error occured.

The last common method of the class will be the \emph{RegisterReflection} me\-thod that is static and that is called automatically when the application initializes. It should register all properties and functions which the component provides by the \emph{RegisterProperty} and \emph{RegisterFunction} methods inherited from the base class. In case of a property the following information must be specified: a type (from PropertyTypes.h), a name (must be unique in a component), a getter (a constant function that returns value of a same type and has no parameters), a setter (a non-constant function with one constant parameter of a same type and no return value), an access flags (a disjunction of the \emph{Reflection::ePropertyAccess} enumerations) and a comment (will be displayed in the editor). A getter and a setter can be simple functions that return or modify a private member variable or they can do a more complex computing. For example the registration of an integer property with a getter and a setter as methods of the class and with a read and write access from the editor looks like:
\begin{verbatim}
RegisterProperty<int32>("IntProp", &ComponentName::GetIntProp,
    &ComponentName::SetIntProp, PA_INIT | PA_EDIT_READ |
    PA_EDIT_WRITE, "This is an integer property.");
\end{verbatim}

After the creation of the class a new component type must be registered in ComponentTypes.h where a new line like \verb/COMPONENT_TYPE(CT_COMPONENT_/ \verb/NAME, ComponentName)/ should be added. Finally the header file where the component is declared must be added to ComponentHeaders.h, for example \verb|#include "../Components/ComponentName.h"/|. Now the component is ready to be added to entities.

\subsection{Adding new entity message types}
\label{sub:entity-newmessages}

If it is necessary to add a new entity message type that can be posted to entities then the new line to EntityMessageTypes.h must be added. It should look like \verb/ENTITY_MESSAGE_TYPE(MESSAGE_NAME, "void OnMessageName(type1,/
\verb/type2, ...)", Params(PT_TYPE1, PT_TYPE2, ...))/ where the first parameter is an enumeration constant, the second one is a declaration of script function that handles the message and the last one is a definition of message parameters where \verb/PT_TYPE/ is from PropertyTypes.h. In case the message has no parameters the \verb/NO_PARAMS/ macro should be provided as the last parameter.

\section{Glossary}

This is a glossary of the most used terms in the previous sections:

\begin{description}
  \item[Entity property] -- a named pair of a getter and a setter function of a specific type with certain access rights
  \item[Entity function] -- a named link to a function with a \emph{Reflection::Property\-FunctionParameters} parameter and certain access rights
  \item[Entity message] -- a structure that stores a message type from EntityMessageTypes.h and message parameters
  \item[Component] -- a class which has registered functions and properties, that can be read and written via their getters and setters, and which can handle received messages
  \item[Entity] -- a compound of one or more components, that provide specific functionalities, represented by an unique ID, it is possible to post a message to it
  \item[Prototype] -- changes of shared property values of this entity are propagated to the linked entities
\end{description}