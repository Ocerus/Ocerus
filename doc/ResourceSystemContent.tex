\noindent\textbf{Namespaces:} ResourceSystem

\noindent\textbf{Classes:} ResourceMgr, Resource, ResourcePtr

\section{Introduction}

Blabla.







\section{Glossary}

This is a glossary of the most used terms in the previous sections:

\begin{description}
  \item[Value type] -- a primitive type (integer or real number, boolean value, enumeration or string) or an object that is copied on an assignment or a passing to or from a function  \item[Reference type] -- an object that is assigned or passed to a function only through a pointer on it
  \item[Object handle] -- an equivalent of a reference in C++ that counts references on an object
	\item[Script class] -- a class defined and implemented in a script file, it is always used as a reference type
	\item[Script function] -- a function defined and implemented in a script file
	\item[Script file] -- one file containing script class and script function definitions, can include a code from other files
	\item[Script module] -- one or more script files connected with include directives, which are managed and builded together and have a common namespace
	\item[Function ID] -- an identification of a script function based on a module name and a function declaration
	\item[Script context] -- an object that wraps a script function calling, it must be prepared with a function ID, executed and released, function arguments can be passed and a return value can be obtained
	\item[Script engine] -- an object that registers C++ classes, global functions, properties etc. for use in a script code and manages script modules and contexts
	\item[Script manager] -- a class that encapsulated a script engine and provides methods for managing script modules and calling script functions
\end{description}

\begin{thebibliography}{9}                                                                                                
\bibitem {angelscript}AngelScript documentation, file /AngelScript/index.html

\end{thebibliography}