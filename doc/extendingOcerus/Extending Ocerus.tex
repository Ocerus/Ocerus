\documentclass[a4paper, 12pt]{report}

\usepackage[USenglish]{babel}
\usepackage[T1]{fontenc}
\usepackage[ansinew]{inputenc}
\usepackage{lmodern} %Type1-font for non-english texts and characters
\usepackage{graphicx}

\newenvironment{titled-itemize}[1]
{
\vspace{5mm}
\noindent\emph{#1}
\begin{itemize}
}
{
\end{itemize}
}

\begin{document}

\pagestyle{empty} %No headings for the first pages.

\title{Guide for extending the Ocerus project}
\author{Lukas Hermann, Ondrej Mocny, Tomas Svoboda, Michal Cevora}

\pagestyle{plain} %Now display headings: headings / fancy / ...

\tableofcontents %Table of contents
\cleardoublepage %The first chapter should start on an odd page.

\chapter{Introduction}

The project Ocerus was designed to be easily extendable on well defined places in code. It allows developers to create even more diverse games. This document serves as a cookbook and shows a sequence of steps that will lead to extension of the Ocerus. There are several ways of extending the Ocerus. Each way will be discussed in separate chapters.
This document only shows how to make things work. If you want to understand how they work, you should see the Design documentation.

\chapter{Components}

\section{blabla}

\chapter{Scripts}

\section{blabla}

\chapter{Resource Types}

\section{blabla}

\chapter{Renderer}

\section{blabla}

%% A small distance to the other stuff in the table of contents (toc)
\addtocontents{toc}{\protect\vspace*{\baselineskip}}

%% The Bibliography 
\begin{thebibliography}{9}
\addcontentsline{toc}{chapter}{Bibliography} %'Bibliography' into toc
\bibitem {angelscript}AngelScript -- http://www.angelcode.com/angelscript

\end{thebibliography}

%% The List of Figures
\clearpage
\addcontentsline{toc}{chapter}{List of Figures}
\listoffigures

%% The List of Tables
\clearpage
\addcontentsline{toc}{chapter}{List of Tables}
\listoftables

\end{document}