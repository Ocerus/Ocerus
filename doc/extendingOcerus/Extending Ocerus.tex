\newcommand{\TitleText}{Ocerus Extension Guide}
\documentclass[a4paper,12pt]{report}
\usepackage[utf8x]{inputenc}

\title{\TitleText}
\author{http://ocerus.sourceforge.net}

\pdfinfo{%
  /Title    (\TitleText)
  /Author   (Michal Cevora, Lukas Hermann, Ondrej Mocny, Tomas Svoboda)
  /Creator  ()
  /Producer ()
  /Subject  ()
  /Keywords ()
}

\usepackage{fullpage} % koneckoncu je to neformalni dokument, takze k certu s akademickou konvenci
\usepackage[pdfborder={0 0 0}, pdfpagelayout=OneColumn, pdftitle={\TitleText}, pdfauthor={Michal Cevora, Lukas Hermann, Ondrej Mocny, Tomas Svoboda}]{hyperref}
\usepackage[USenglish]{babel}
\usepackage[all]{hypcap}
\usepackage{listings}
\usepackage{graphicx}
\usepackage{wrapfig}


\begin{document}

\maketitle
\clearpage

{
	\renewcommand\thepage{}
	\tableofcontents
	\clearpage
}

\setcounter{page}{1}



\chapter{Introduction}

The project Ocerus was designed to be easily extendable on well defined places in code. It allows developers to create even more diverse games. This document serves as a cookbook and shows a sequence of steps that will lead to the extension of the Ocerus. There are several ways of extending the Ocerus. Each way will be discussed in separate chapters.

This document only shows how to make things work. If you want to understand how they work, you should see the Design documentation.

\chapter{Components}

Every game object is represented by an entity which is a compound of components that provide it with various functionalities. A component can have several properties (and functions), which can be read or written (called) via their getters and setters (or functions themselves), and which are accessible through their unique name. It can also react to sent messages such as an initialization, a drawing, a logic update etc. by its own behavior.

The Ocerus provides some predefined components that provide basic features such as a graphic representation,  physics, scripts etc. It is possible to define new components. They can add new functionality to the game entities. Similar result can be achieved by using scripts but the logic in components is not limited to predefined script functions and runs faster. In the following sections, there will be a sequence of steps that are necessary to create and use a new component. 

There will be also an example of a simple component that will make an entity to move up and down. It would probably be better to implement this particular functionality by a script but it will serve well as an example.

\section{Creating a component class}
The first thing that needs to be done is creating a new component class. Each component class is defined in a separate header file. These files are located in \emph{EntitySystem/Components}.

The header file of a component must be defined in the \emph{EntityComponents} namespace and must publicly inherit from the  \emph{RTTIGlue$<$[Component Name], Component$>$}. So the example component class (named \emph{ExampleComp}) would look like this:

\footnotesize 
\begin{verbatim}
namespace EntityComponents
{
  class ExampleComp : public RTTIGlue<ExampleComp, Component>
  {
  }
}
\end{verbatim}
\normalsize

There are few methods in this class that should be overridden: \emph{Create}, \emph{Destroy}, \emph{HandleMessage} and \emph{RegisterReflection}. \emph{Create} and \emph{Destroy} methods are called after the creation and before the destruction of the component, respectively. \emph{HandleMessage} and \emph{RegisterReflection} methods will be described in later sections.

After that, the component type must be registered in the \emph{\_ComponentTypes.h} header file (located in \emph{EntitySystem/Components}). To do so, this line should be added:
\footnotesize 
\begin{verbatim}
COMPONENT_TYPE(ExampleComp)
\end{verbatim}
\normalsize

Finally, the header file with the component class must be included in the \emph{\_Component\-Headers\-.h} header file.

\section{Adding component properties}

The component properties are represented by the class member variables. But just defining a class member variable is not enough to turn it into a regular property. It needs to be registered in the RTTI. Registering properties is done in the static method \emph{void RegisterReflection()}. It is automatically called during the initialization process of the \emph{EntitySystem}. In this method, there should be called the \emph{RegisterProperty} function. Usage of that function looks like this:

\footnotesize 
\begin{verbatim}
RegisterProperty<Property type>("Property name", &GetterMethod, 
  &SetterMethod, Access flag, "Property description");
\end{verbatim}
\normalsize

\begin{itemize}
\item
The Property type is obviously the data type of the property but is must be registered in the \emph{PropertyTypes.h} header file (located in \emph{Utils/Properties}). All regular property types are already registered there. There is also a short description of how to register new property types.

\item
The property must have getter and setter methods defined. The methods signatures must look like this:

\footnotesize 
\begin{verbatim}
[Property type] GetProp(void) const
void SetProp([Property type] value)
\end{verbatim}
\normalsize

Addresses of these methods must be passed to the \emph{RegisterProperty} function.

\item
The Access flag defines ways how the property can be accessed from the editor or a script or whether it should be saved/loaded to/from a stream. The list of all flag states is in the \emph{PropertyAccess.h} header file (located in \emph{Utils/Properties}).
\end{itemize}

Also component functions can be registered so they can be called from a script. For that, there is the \emph{RegisterFunction} function. Usage of that function is similar to the \emph{RegisterProperty} and looks like this:

\footnotesize 
\begin{verbatim}
RegisterFunction("Function Name", &Function, Access flag, "Function description");
\end{verbatim}
\normalsize

Back to the example component. Let's say it will have three properties and one function. First property for the amplitude, second for the period of moving, third for the timer and a function for resetting the timer. All three properties will be of the \emph{float32} data type. Setters and getters must be defined as well as the \emph{RegisterReflection} function. 

The example component needs to access and change the position of the entity. The position is defined in the \emph{Transform} component. To ensure that the entity has the \emph{Transform} component, the dependency must be added. For that purpose the \emph{AddDependency} function serves. It receives a component type as a parameter (component type is defined as \emph{CT\_[Component class name]}). It will also force the dependent component to initialize later than the other component.

So the example component class will look like this: 

\footnotesize 
\begin{verbatim}
namespace EntityComponents
{
  class ExampleComp : public RTTIGlue<ExampleComp, Component>
  {
  public:
    // Creation of the component. Sets all properties to 0.
    virtual void Create(void) {mAmplitude = mPeriod = mTimer = 0;}
    // Destruction of the component. Nothing needs to be destructed.
    virtual void Destroy(void) {}
    
    // Function for resetting the timer.
    void ResetTimer(void)
    {
       mTimer = 0;
    }

    static void RegisterReflection(void)
    {
      // Register the Amplitude property
      RegisterProperty<float32>("Amplitude", &GetAmplitude,
        &SetAmplitude, PA_FULL_ACCESS, "Amplitude of moving.");

      // Register the Period property
      RegisterProperty<float32>("Period", &GetPeriod, 
        &SetPeriod, PA_FULL_ACCESS, "Period of moving.");

      // Register the Timer property
      RegisterProperty<float32>("Timer", &GetTimer, 
        &SetTimer, PA_FULL_ACCESS, "Time from the start.");

      // Register the ResetTimer function
      RegisterFunction("ResetTimer", &ResetTimer,
        PA_SCRIPT_WRITE, "Resets the timer.");

      // Add dependency on the Transform component
      AddComponentDependency(CT_Transform);
    }

    // Getter and setter for the Amplitude property
    float32 GetAmplitude(void) const { return mAmplitude; }
    void SetAmplitude(float32 value) { mAmplitude = value; }

    // Getter and setter for the Period property
    float32 GetPeriod(void) const { return mPeriod; }
    void SetPeriod(float32 value) { mPeriod = value; }

    // Getter and setter for the Timer property	
    float32 GetTimer(void) const { return mTimer; }
    void SetTimer(float32 value) { mTimer = value; }

  private:
    // Definition of member variables that are used as properties.
    float32 mAplitude;
    float32 mPeriod;
    float32 mTimer;
  }
}
\end{verbatim}
\normalsize

\section{Entity messages}

Entities can receive messages. They are sent by the engine when some event occurs. It can be an initialization of the entity, an update in physics, a pressed key etc. Messages can be also sent from a script.

A message that is sent to an entity is distributed to all its components. It can also carry parameters.

\subsection{Message types}
Message types are defined in the \emph{EntityMessageTypes.h} header file (located in \emph{EntitySystem/EntityMgr}). New message types can be easily added if needed. The declaration of the message type is done by macros and looks like this:

\footnotesize 
\begin{verbatim}
ENTITY_MESSAGE_TYPE([Name], "[Script function]", [Parameters] )
\end{verbatim}
\normalsize

\begin{itemize}
\item
Name is simply the name of the message type. Must be unique.
\item
If an entity has a script component, then when it receives a message, a function in script is called. And that function name and signature is defined by the second parameter in the \emph{ENTITY\_MESSAGE\_TYPE} macro.
\item
The third parameter defines parameters that can be added to the message. When no parameter is needed, then \emph{NO\_PARAMS} macro should be used. Otherwise, the \emph{Params} macro should be used. It accepts property types as parameters. So a message with  \emph{float32} and  \emph{int32} parameters will have \emph{Params(PT\_FLOAT32, PT\_INT32)}.

Note that the parameters must correspond to the script function signature defined in the second parameter of the \emph{ENTITY\_MESSAGE\_TYPE} macro. So it could look like this: 

\footnotesize 
\begin{verbatim}
ENTITY_MESSAGE_TYPE(HELLO, "void Hello(float32, int32)",
  Params(PT_FLOAT32, PT_INT32) )
\end{verbatim}
\normalsize

\end{itemize}

\subsection{Handling messages}

 The handling of messages is done by the \emph{HandleMessage} method of the component class. It accepts the \emph{EntityMessage} as a parameter. The \emph{EntityMessage} contains the message type and message parameters. 

The \emph{HandleMessage} method must return result: \emph{RESULT\_OK} if the message was accepted, \emph{RESULT\_IGNORED} when the message was ignored and \emph{RESULT\_ERROR} if an error occurred.

The example component, which makes an entity to move up and down, could have the \emph{HandleMessage} method implemented like this:

\footnotesize 
\begin{verbatim}
EntityMessage::eResult EntityComponents::ExampleComp::
  HandleMessage( const EntityMessage& msg )
{
  // Decide the type of the received message
  switch (msg.type)
  {
  // Called when the component is initialized
  case EntityMessage::INIT:
    {
      // Get transform component
      Component* transform = gEntityMgr.
        GetEntityComponentPtr(GetOwner(), CT_Transform);
      // Get position property from component
      // mStartPosition must be declared as a member variable
      mStartPosition = transform->GetProperty("Position").GetValue<Vector2>();

      return EntityMessage::RESULT_OK;
    }
  // Called when the logic is updated
  case EntityMessage::UPDATE_LOGIC:
    {
      // Get the first message parameter
      // Here it is the time from the last logic update
      float32 delta = *msg.parameters.GetParameter(0).GetData<float32>();

      if ((delta <= 0) || (mPeriod <= 0))
        return EntityMessage::RESULT_OK;

      mTimer += delta;

      // Calculate an offset from the start position
      float32 offset = mAmplitude * 
        MathUtils::Sin(mTimer / mPeriod * 2 * MathUtils::PI);

      // Get transform component
      Component* transform = gEntityMgr.
        GetEntityComponentPtr(GetOwner(), CT_Transform);

      Vector2 pos = mStartPosition;
      pos.y += offset;

      // Set new value to the position property
      transform->GetProperty("Position").SetValue(pos);

      return EntityMessage::RESULT_OK;
    }
    default:
      return EntityMessage::RESULT_IGNORED;
  }
}
\end{verbatim}
\normalsize

\section{Conclusion}

After above described steps, the new component is ready to be used. It will appear in the editor and component properties and functions will be accessible by a script.

Note that most functionalities are easier to implement with scripts. The main advantage (and also the main disadvantage) of creating a new component is that it is defined in the source code. The new component can directly access the engine but the system has to be recompiled with every change to the component. 

The scripts, on the other hand, can be changed real-time, when the editor is running. It makes developing scripts much easier. The disadvantage of scripts is that it has only a limited list of functions. But that list can be extended, see chapter \emph{\nameref{chap:script_system}}.

\chapter{Script system}
\label{chap:script_system}

The Script system provides quite a powerful tool for programming the game logic without having to modify and compile the source code. Even if there are hundreds of registered functions, classes, operators etc., it could be useful to register some more. The Script system is prepared for this so it is not difficult to register some new stuff.

The following section will show a simple example of how to register a new class with methods in the Script system. It will be described only briefly. For more details, please see the AngelScript documentation \cite{angelscript}. Especially chapter \emph{Using AngelScript/Registering the application interface}.

\section{Registering a new class}

All new classes and functions should be registered in the \emph{RegisterAllAdditions} function in the \emph{ScriptRegister.cpp} file (located in \emph{ScriptSystem}).

Registering of new class (named \emph{NewClass}) looks like this:

\footnotesize 
\begin{verbatim}
class NewClass
{
  ...
};

r = engine->RegisterObjectType("NewClass", sizeof(NewClass),
 asOBJ_VALUE | asOBJ_APP_CLASS_C); OC_SCRIPT_ASSERT();
\end{verbatim}
\normalsize

\begin{itemize}
\item
The method returns an integer typed result. The name of the variable, the result is being assigned to, should be \emph{r}, because it is then checked for an error in the \emph{OC\_SCRIPT\_ASSERT()} macro.

\item
The first parameter of the method defines the name of the class in a script.

\item
The second parameter defines the size of the class in bytes.

\item
The last parameter is a flag defining the behavior. For details see the AngelScript documentation \cite{angelscript}. This particular combination means that the class is handled as object value (not reference) and needs to be constructed and destructed by constructor and destructor.
\end{itemize}

\subsection{Class Methods}
Class methods should be registered like this:

\footnotesize 
\begin{verbatim}
class NewClass
{
  int32 ClassMethod(float32 f)
  {
    // Do something
  }
  ...
};

r = engine->RegisterObjectMethod("NewClass", "int32 class_method(float32)", 
  asMETHODPR(NewClass, ClassMethod, (float32), int32), asCALL_THISCALL);
  OC_SCRIPT_ASSERT();
\end{verbatim}
\normalsize

\begin{itemize}
\item
The first parameter is the name of the methods class.
\item
The second parameter defines name and signature of the method in a script.
\item
The third parameter defines type and C++ name of the function. When the function is method of some class (as in this example), then \emph{asMETHODPR} macro is used with parameters: name of class, name of method, parameter types and return type.
\item
The last parameter defines how the function should be called. \emph{asCALL\_THISCALL} means that it should be called as a class method. Details in the AngelScript documentation\cite{angelscript}.
\end{itemize}

\subsection{Constructor and Destructor}

If a constructor or destructor is needed they shall be registered the following way:

\footnotesize 
\begin{verbatim}
void Constructor(NewClass *memory)
{
  // Initialize the pre-allocated memory by calling the
  // object constructor with the placement-new operator
  new(memory) NewClass();
}

void Destructor(NewClass *memory)
{
  // Finalize the memory by calling the object destructor
  memory->~NewClass();
}

// Register the constructor
r = engine->RegisterObjectBehaviour("NewClass", asBEHAVE_CONSTRUCT, "void f()",
  asFUNCTION(Constructor), asCALL_CDECL_OBJLAST); OC_SCRIPT_ASSERT();

// Register the destructor
r = engine->RegisterObjectBehaviour("NewClass", asBEHAVE_DESTRUCT, "void f()",
  asFUNCTION(Destructor), asCALL_CDECL_OBJLAST); OC_SCRIPT_ASSERT();

\end{verbatim}
\normalsize

\section{Global function}
Similarly, a global function can be added.

\footnotesize 
\begin{verbatim}
r = engine->RegisterGlobalFunction("float32 NewGlobFnc(int32)", 
  asFUNCTIONPR(GlobFnc, (int32), float32), asCALL_CDECL); OC_SCRIPT_ASSERT();
\end{verbatim}
\normalsize

The parameters are the same as in the \emph{RegisterObjectMethod} method, only the first one is missing.

 It is also possible to register enumerations (\emph{RegisterEnum, RegisterEnumValue}) and typedefs for primitive types (\emph{RegisterTypedef}). The usage is quite easy and can be found in the AngelScript documentation\cite{angelscript}.

\section{Usage of the new class in a script}
After the new class was registered, it can be used in a script just like in normal C++ code.
\footnotesize 
\begin{verbatim}
  NewClass c;
  c.class_method(0);
\end{verbatim}
\normalsize

\section{Conclusion}
Scripts should be used for most of the game logic including AI, GUI control, input etc.
Registering new functions to the Script system can add some new functionality but it could also help performance. The logic in scripts runs slower than the logic in the source code. For example the pathfinding could be too slow, when implemented in a script. A faster solution would be to define functions that compute the pathfinding in the source code and then register the functions that return the result to the Script system.


\chapter{Resource types}

All resources in the Ocerus are managed by the Resource manager. Each resource has its type. There are some basic resource types defined in the Resource manager but sometimes it would be nice to define some new.

New resource types are defined as a class derived from the \emph{Resource} class. It is necessary to add a new resource type to the \emph{eResourceType} enum and a resource type name to the \emph{ResourceTypeNames} string array. Both are defined in the \emph{ResourceType.h} header.

The new resource class has to override some methods:

\begin{itemize}
\item
Constructor and destructor.
\item
Static \emph{CreateMe()} method that returns a special shared pointer (\emph{ResourceSystem\::ResourcePtr}) that points to the new resource class instance. This pointer allows type-checked casting to specific resource pointers. So it should look like this:

\footnotesize 
\begin{verbatim}
ResourceSystem::ResourcePtr NewResource::CreateMe()
{
  return ResourceSystem::ResourcePtr(new NewResource());
}
\end{verbatim}
\normalsize

\item
Static GetResourceType() method that returns the \emph{eResourceType} of the resource.

\footnotesize 
\begin{verbatim}
static ResourceSystem::eResourceType GetResourceType() 
{ 
  return ResourceSystem::RESTYPE_NEWRESOURCE; 
}
\end{verbatim}
\normalsize

\item
Protected abstract method \emph{LoadImpl}. This method is called when the resource is being loaded. It should parse the data from the file stream into something useful. It should use the \emph{DataContainer} class and the \emph{GetRawInputData} method for getting data from the source file.

So the \emph{LoadImpl} method will look like this:

\footnotesize 
\begin{verbatim}
size_t NewResource::LoadImpl()
{
  DataContainer dc;
  size_t dataSize = 0;

  // loads data from the file to the data container
  // returns true if successful
  if (GetRawInputData(dc))	
  {
    dataSize = dc.GetSize();

    // get pointer to data byte stream
    uint8* data = dc.GetData();

    // parse the data
    // save the data into own data structure

    dc.Release();
  }
  return dataSize;
}
\end{verbatim}
\normalsize

The \emph{LoadImpl} method must return the size of the resource in the memory in bytes. If 0 is returned, it means that the resource failed to load.

\item
Protected abstract method \emph{UnloadImpl}. This method is called when the resource is being unloaded. It should release all the data the resource was using.

\item
The resource class can have additional methods for accessing the resource type specific structures. Those method should always begin with calling the \emph{EnsureLoaded()} method. It ensures that the resource is loaded (the \emph{LoadImpl} method was called).

\end{itemize}

Defining new resource types is necessary when adding new functionality that requires data from external files. For example, a new component that adds sounds to an entity would need a new resource type. The component should have a \emph{ResourcePtr} typed property, through which it could access the resource data, and it should define and export functions that will allow to play the sound from a script.

\chapter{Renderer}

The Ocerus uses an OpenGL renderer by default. But it is possible to implement an own renderer that uses a different graphic library.

The only part of code, which uses the OpenGL, is the \emph{OglRenderer} class. It derives from the \emph{GfxRenderer} class and implements its abstract methods. There is about 20 methods that needs to be implemented by the new renderer. They are described in the Doxygen documentation.

After the new renderer class is implemented, the initialization of the renderer must be changed. In the \emph{Application.cpp} file (located in \emph{Core}), the line:

\begin{verbatim}
GfxSystem::GfxRenderer::CreateSingleton<GfxSystem::OglRenderer>();
\end{verbatim}
\normalsize

\noindent should be replaced by:

\begin{verbatim}
GfxSystem::GfxRenderer::CreateSingleton<GfxSystem::NewRenderer>();
\end{verbatim}
\normalsize

%% A small distance to the other stuff in the table of contents (toc)
\addtocontents{toc}{\protect\vspace*{\baselineskip}}

%% The Bibliography 
\begin{thebibliography}{9}
\addcontentsline{toc}{chapter}{Bibliography} %'Bibliography' into toc
\bibitem {angelscript}AngelScript documentation -- 

http://www.angelcode.com/angelscript/sdk/docs/manual/index.html

\end{thebibliography}

\end{document}